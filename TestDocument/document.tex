\documentclass{article}
\usepackage{adjustbox}
\usepackage{float}
\usepackage{textcomp}
\usepackage{graphicx}
\graphicspath{{images/}}
\usepackage{booktabs}
\usepackage{color}
\usepackage{verbatim}
\usepackage{listings}
\usepackage{underscore}
\setcounter{secnumdepth}{5}
\usepackage[bookmarks=true]{hyperref}
\author{Roberto Clapis (841859), Erica Stella (854443)} 
\date{\today}
\title{Politecnico di Milano
	\\A.A. 2015\@-\@2016
	\\Software Engineering 2: ``myTaxiService''
	\\\textbf{I}ntegration textbf{T}est textbf{P}lan \textbf{D}ocument}
\hypersetup{
	pdftitle={Integration Test Plan Document},    % title
	pdfauthor={Roberto Clapis, Erica Stella},                     % author
	pdfsubject={Integration Test Plan Document},                        % subject of the document
	pdfkeywords={TeX, LaTeX, taxi, ITPD, SoftwareEngineering2}, % list of keywords
	colorlinks=true,       % false: boxed links; true: colored links
	linkcolor=black,       % color of internal links
	citecolor=blue,       % color of links to bibliography
	filecolor=black,        % color of file links
	urlcolor=purple,        % color of external links
}
\begin{document}
\maketitle
\begin{center}
	\includegraphics{polimi-logo}
\end{center}
\clearpage
\tableofcontents
\clearpage
\section{Introduction}
\subsection{Revision History}
%Record all revisions to the document
\subsection{Purpose and Scope}
%State the purpose and scope of the document
\subsection{List of Definitions and Abbreviations}
\subsection{List of Reference Documents}
%List all reference documents, for instance: the project description, the RASD, the Design document, the documentation of any tool you plan to use for testing
\section{Integration Strategy}
\subsection{Entry Criteria}
%Specify the criteria that must be met before integration testing of specific elements may begin (e.g., functions must have been unit tested).
\subsection{Elements to be Integrated}
%Identify the components to be integrated, refer to your design document to identify such components in a way that is consistent with your design 
\subsection{Integration Testing Strategy}
Describe the integration testing approach (top-down, bottom-up, functional groupings, etc.) and the rationale for the choosing that approach
\subsection{Sequence of Component/Function Integration}
%NOTE: the structure of this section may vary depending on the integration strategy you select in Section 2.3. Use the structure proposed below as a non mandatory guide
\subsubsection{Software Integration Sequence}
%For each subsystem: identify the sequence in which the software components will be integrated within the subsystem. Relate this sequence to any product features/functions that are being built up
\subsubsection{Subsystem Integration Sequence}

\end{document}