\documentclass{article}
\usepackage{adjustbox}
\usepackage{float}
\usepackage{textcomp}
\usepackage{graphicx}
\graphicspath{{images/}}
\usepackage{booktabs}
\usepackage{color}
\usepackage{verbatim}
\usepackage{listings}
\usepackage{underscore}
\setcounter{secnumdepth}{5}
\usepackage[bookmarks=true]{hyperref}
\author{Roberto Clapis (841859), Erica Stella (854443)} 
\date{\today}
\title{Politecnico di Milano
	\\A.A. 2015\@-\@2016
	\\Software Engineering 2: ``myTaxiService''
	\\\textbf{I}ntegration \textbf{T}est \textbf{P}lan \textbf{D}ocument}
\hypersetup{pdftitle={Integration Test Plan Document},    % title
	pdfauthor={Roberto Clapis, Erica Stella},                     % author
	pdfsubject={Integration Test Plan Document},                        % subject of the document
	pdfkeywords={TeX, LaTeX, taxi, ITPD, SoftwareEngineering2}, % list of keywords
	colorlinks=true,       % false: boxed links; true: colored links
	linkcolor=black,       % color of internal links
	citecolor=blue,       % color of links to bibliography
	filecolor=black,        % color of file links
	urlcolor=purple,        % color of external links
}
\begin{document}
\maketitle
\begin{center}
	\includegraphics{polimi-logo}
\end{center}
\clearpage
\tableofcontents
\clearpage
\section{Introduction}
\subsection{Revision History}
%Record all revisions to the document
\subsection{Purpose and Scope}
%State the purpose and scope of the document
This document describes the Integration Test Plan for the myTaxiService application. It provides a plan referring to how the various components of the software architecture described in the Design Document will be integrated for testing. 
\subsection{List of Definitions and Abbreviations}
\subsection{List of Reference Documents}
%List all reference documents, for instance: the project description, the RASD, the Design document, the documentation of any tool you plan to use for testing
\begin{itemize}
	\item The document with myTaxiService's description
	\item myTaxiService's RASD
	\item myTaxiService's Design Document
\end{itemize}
\section{Integration Strategy}
\subsection{Entry Criteria}
%Specify the criteria that must be met before integration testing of specific elements may begin (e.g., functions must have been unit tested).
Before the integration testing, each single module must have been tested to verify its correct functioning according to its specifications.
\subsection{Elements to be Integrated}
%Identify the components to be integrated, refer to your design document to identify such components in a way that is consistent with your design 
According to the Design Document, the components to be integrated are:
\begin{itemize}
	\item Accounts
	\item Active Reservations and Requests
	\item DBConnector
	\item APIBackend
	\item WebpageCreator
	\item NotificationModule
	\item HttpHandler
	\item ClientUI
	\item DriverUI
	\item AdminUI
\end{itemize}
\subsection{Integration Testing Strategy}
%Describe the integration testing approach (top-down, bottom-up, functional groupings, etc.) and the rationale for the choosing that approach
The decided testing approach is bottom-up. %TODO inserire motivazione del perchè
\subsection{Sequence of Component/Function Integration}
%NOTE: the structure of this section may vary depending on the integration strategy you select in Section 2.3. Use the structure proposed below as a non mandatory guide
\subsubsection{Software Integration Sequence}
%For each subsystem: identify the sequence in which the software components will be integrated within the subsystem. Relate this sequence to any product features/functions that are being built up
\subsubsection{Subsystem Integration Sequence}
%Identify the order in which subsystems will be integrated. If you have a single subsystem, 2.4.1 and 2.4.2 are to be merged in a single section. You can refer to Section 2.2 of the test plan example [1] as an example of what we expect
\section{Individual Steps and Test Description}
%For each step of the integration process identified above, describe the type of tests that will be used to verify that the elements integrated in this step perform as expected. Describe in general the expected results of the test set. You may refer to Chapter 3 and Chapter 4 of the test plan example [1] as an example of what we expect. (NOTE: this is not a detailed description of test protocols. Think of this as the test design phase. Specific protocols will be written to fulfill the goals of the tests identified in this section.)
\subsection{Test case specifications}
	WebServer \rightarrow DBMS
	\begin{tabular}{*{2}{c}}
		\toprule
		Test Case identifier & I1T1 \rightarrow I1TN\\
		Test Items & DBConnector \rightarrow Accounts \\
		Input Specification & Queries to manipulate (creation modification and deletion) of accounts\\
		Output Specification & The actual modifications intended\\
		Environmental Needs & DBMS Driver\\
		\bottomrule
	\end{tabular}
	\begin{tabular}{*{2}{c}}
		\toprule
		Test Case identifier & I2T1\\
		Test Items & DBConnector \rightarrow Active Reservations and Requests \\
		Input Specification & Queries to place/accept/delete reservations and requests, in every possible order of execution\\
		Output Specification & The actual modifications intended and the rejection of the invalid requests\\
		Environmental Needs & DBMS Driver\\
		\bottomrule
	\end{tabular}

	WebServer 
	\begin{tabular}{*{2}{c}}
		\toprule
		Test Case identifier & IT1\\
		Test Items & DBConnector \rightarrow Accounts \\
		Input Specification & Queries to manipulate (creation modification and deletion) of accounts\\
		Output Specification & The actual modifications intended\\
		Environmental Needs & DBMS Driver\\
		\bottomrule
	\end{tabular}
	
	
	
	
	
	
	
	
	
	
	
	\subsection{Test procedures}
	\section{Tools and Test Equipment Required}
	\begin{tabular}{*{2}{c}}
		\toprule
		Test Case identifier & I1T1\\
		Test Items & \rightarrow\\
		Input Specification & \\
		Output Specification & \\
		Environmental Needs & \\
		\bottomrule
	\end{tabular}
	\begin{tabular}{*{2}{c}}
		\toprule
		Test Case identifier & \\
		Test Items & \\
		Input Specification & \\
		Output Specification & \\
		Environmental Needs & \\
		\bottomrule
	\end{tabular}
	%Identify all tools and test equipment needed to accomplish the integration. Refer to the tools presented during the lectures. Explain why and how you're going to use them. Note that you may also use manual testing for some part. Consider manual testing as one of the possible tools you have available.
	\section{Program Stubs and Test Data Required}
	%Based on the testing strategy and test design, identify any program stubs or special test data required for each integration step.
	\section{Appendix}
	Appendix for Roberto Clapis\\
	Work hours: 20
	\begin{center}
		Software Used:\\
		\-\\
		\begin{tabular}{*{2}{c}}
			\toprule
			Task & Software \\
			\midrule
			Edit \LaTeX\ Source & Vim\\
			Edit Graphs Sources & Vim\\
			Edit sources for Sequence Diagrams & Vim\\
			Convert Sequence Diagrams to images & Quick Sequence Diagram Editor\\
			Generate and Raster directed graphs& Dot\\
			Generate and Raster undirected graphs& Fdp\\
			General images mangling and cropping & ImageMagick \& Shotwell\\
			Convert \LaTeX\ source to PDF & \LaTeX\-MK\\
			Spell Check & Aspell \\
			\LaTeX\ Check & LaCheck\\
			\bottomrule
		\end{tabular}
	\end{center}
	\-\\
	\-\\
	Appendix for Erica Stella\\
	Work hours: 20
	\begin{center}
		Software Used:\\
		\-\\
		\begin{tabular}{*{2}{c}}
			\toprule
			Task & Software \\
			\midrule
			Edit \LaTeX\ Source & TexStudio\\
			Convert \LaTeX\ source to PDF & \LaTeX\-MK\\
			Edit sources for Sequence Diagrams & Quick Sequence Diagram Editor\\
			Convert Sequence Diagrams to images & Quick Sequence Diagram Editor\\
			\bottomrule
		\end{tabular}
	\end{center}

	\end{document}

