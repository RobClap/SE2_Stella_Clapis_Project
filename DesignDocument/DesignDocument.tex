\documentclass{article}
\usepackage{float}
\usepackage{textcomp}
\usepackage{graphicx}
\graphicspath{{images/}}
\usepackage{booktabs}
\usepackage{color}
\usepackage{verbatim}
\usepackage{listings}
\usepackage{underscore}
\setcounter{secnumdepth}{5}
\usepackage[bookmarks=true]{hyperref}
\author{Roberto Clapis (841859), Erica Stella (854443)} 
\date{\today}
\title{Politecnico di Milano
		\\A.A. 2015\@-\@2016
		\\Software Engineering 2: ``myTaxiService''
		\\\textbf{D}esign \textbf{D}ocument}
		\hypersetup{pdftitle={Design Document},    % title
		pdfauthor={Roberto Clapis, Erica Stella},                     % author
		pdfsubject={Design Document},                        % subject of the document
		pdfkeywords={TeX, LaTeX, taxi, DD, SoftwareEngineering2}, % list of keywords
		colorlinks=true,       % false: boxed links; true: colored links
		linkcolor=black,       % color of internal links
		citecolor=blue,       % color of links to bibliography
		filecolor=black,        % color of file links
		urlcolor=purple,        % color of external links
}
\begin{document}
\maketitle
\begin{center}
	\includegraphics{polimi-logo}
\end{center}
\clearpage
\tableofcontents
\clearpage

\section{Introduction}
\subsection{Purpose}
The purpose of the Design Document is to provide documentation in order to aid the development of myTaxiService's system by providing a description of how it should be built and how its components are expected to interact with each other.
\subsection{Scope}
This Design Document is intended to explain the design and architecture of myTaxiService, a new application that will provide an easy way to access the taxi service in a city. It describes the system both from a software and hardware point of view, in order to clarify the system's structure and how it accomplishes its functionalities. 
\subsection{Definitions, Acronyms, Abbreviations}
\paragraph{Definitions}
\begin{itemize}
	\item \textit{End users:} this category comprises all those who use the application\footnotemark: administrators, taxi drivers, logged in users and guests.
\end{itemize}
\footnotetext{For their definition we refer to the RASD's section 1.6}
\paragraph{Acronyms}
\begin{itemize}
	\item \textit{UI:} user interface through which the end users can interact with the application.
	\item \textit{DB:} Database.
	\item \textit{DBMS:} Database Management System.
\end{itemize}
\subsection{Reference Documents}
\begin{itemize}
	\item \href{run:./external_references/assignments.pdf}{Document} with the assignment for the project
	\item \href{run:./external_references/Rasd.pdf}{RASD} for myTaxiService
	\item \href{run:./external_references/DDTOC.pdf}{Template} for the Design Document
	\item \href{run:./external_references/IEEESoftwareDesignDescriptions.pdf}{IEEE standard} for Software Design Document
	\item The \href{run:./external_references/IEEEArchitectureDescription.pdf}{IEEE standard} for architecture descriptions
\end{itemize}
\subsection{Document Structure}
The following parts of this document are structured in 3 sections: architectural design, user interface design and requirements traceability. The architectural design section describes the software and hardware components of the system and their interactions. The user interface design section which refers to the ``User Interfaces'' subsection of the RASD\@. The requirements traceability section that explains how the proposed design meets the requirements that have been defined in the RASD\@.
\clearpage
\section{Architectural Design}
\subsection{Overview}
The system to be developed, as mentioned before, will be used to provide an easy access to a taxi service. Therefore, its main functionalities, that will have to be supported by the design and architecture, are: the storage of the taxi drivers' and clients' accounts, the computation of the taxi queue of each zone and the handling of requests and reservations. Furthermore, the system will have to comply to quality of service attributes as specified in the RASD\@.
\subsection{High Level Components and Their Interaction}
myTaxiService's system is composed by three main components: DBMS, Web Server and client application. The client application provides the UI through which end users can access the application's services. These requests are forwarded to the Web Server which is in charge of providing a response, eventually querying the Database in the DBMS for information. The Web Server is also responsible for answering to the API calls coming from external applications and notifying the end users for particular events, like when a request is accepted by a taxi driver. The DBMS stores all the information of the end users's accounts, the active requests and reservations. %TODO anche lo stato delle code è nel db giusto? NOPE, viene ricalcolato ogni volta che serve saperlo, se no potremmo avere race conditions e stati incoerenti del server, tutta roba in più da gestire
\subsection{Component View}
\begin{figure}[H]
	  \makebox[\textwidth][c]{\includegraphics[width=1.4\textwidth]{Component}}%
\end{figure}
	  Meanwhile the DBMS and the ClientSide application are pretty much self explanatory the WebServer part requires some details:
	  \begin{itemize}
			  \item The Webpage Creator is the responsible for both the Mobile App and Web App responses. The httpHandler will recognise the request by the parameters and ask the Webpage Creator to either create the full Html page (in the case of the Web App) or just send a partially created page that only contains the dynamic data that the Mobile App will then include in its interface.
			  \item The API Backend will be called by the httpHandler and will respond with Only the data requested in a JSON format
	  \end{itemize}
\subsection{Deployment View}
\begin{figure}[H]
	  \makebox[\textwidth][c]{\includegraphics[width=1.3\textwidth]{Deployment}}%
\end{figure}
		This is just a pretty standard configuration with DMZ, Internal network, double firewall and untrusted network, see more in section 2.8
\subsection{Runtime View}
Some details:
\begin{itemize}
	\item Https negotiations are omitted in order to preserve readability, but they would all be between the apps (Web or Mobile)  and the httpHandler
	\item Queues are not represented in the database, but they are all re-computed with a query that selects the taxis that are active, are in the desired area and returns them sorted by the time they were marked as active.
\end{itemize}
%TODO sequence diagrams
\subsection{Component Interfaces}
\paragraph{DBInteractions}
This interface encapsulates and exposes all the operations the Web Server needs to interact with the DB\@. The operations are divided in 2 categories based on the DB's part they manage:
\begin{itemize}	
	\item \textit{Accounts managing and access:} this interface exposes methods to manage the stored end users accounts.\\	
	
		%TODO QUESTA COSA ESCE, dobbiamo trovare un modo di centrarla
	\begin{figure}[H]	
		\makebox[\textwidth][c]{\begin{tabular}{*{3}{c}}
			\toprule
			Method & Parameters required & Notes \\
			\midrule
			addAccount & account type and credentials & adds an account to the DB of the specified type\\
			modifyAccount & user ID, attribute to modify, and value & updates the attribute specified to the new value\\ 
			deleteAccount & user ID & deletes the account specified by the ID \\
			getAccount & user ID & returns the account corresponding to the ID with all its attributes \\
			getAllAccountsOfOneType & type of account & returns all the accounts of the specified type \\
			\bottomrule
		\end{tabular}	
	}
	\end{figure}
	\item \textit{Operations creations and querying:} this interface exposes methods to manage and retrieve requests and reservations.\\	
	%TODO forse aggiungere gli user ID nei parametri yes
	\begin{figure}[H]	
		\makebox[\textwidth][c]{\begin{tabular}{*{3}{c}}
			\toprule
			Method & Parameters required & Notes \\
			\midrule
			addRequest & starting location's zone, destination's zone, user's ID & adds a new request to the DB from the specified user and returns its ID to the caller\\ 
			deleteRequest & request ID & deletes an existing request from the DB\\ 
			toggleStateOfTheRequest & request ID & changes the state of the request from ``not accepted'' to ``accepted''\\
			getRequest & request ID & returns the request specified by its ID with all its attributes \\
			getAllTheRequestsOfAnUser & user ID & returns all the active requests of the specified user \\
			addReservation & starting locations's zone, destination's zone, meeting time, user ID & adds a new reservation to the DB from the specified user\\
			deleteReservation & reservation ID & deletes an existing reservation from the DB\\
			toggleStateOfTheReservation & reservation ID & changes the state of the reservation from ``not accepted'' to ``accepted''\\
			getReservation & reservation ID & returns the reservation specified by its ID with all its attributes\\
			getAllTheReservationsOfAnUser & user ID & returns all the active reservations of the specified user \\ 
			\bottomrule
		\end{tabular}	
	}\end{figure}
\end{itemize}
\paragraph{Http/s Request}
This interface exposes  %TODO finire frase

\paragraph{Response creator}
This interface exposes all the functionalities end users require to access the services offered by myTaxiService. 
In the following list the methods are divided by type of end user that needs them.
\begin{itemize}
	\item \textit{Functionalities exploited by guests:}	\\
	\begin{figure}[H]	
		\makebox[\textwidth][c]{\begin{tabular}{*{3}{c}}
				\toprule
				Method & Parameters required & Notes \\
				\midrule
				searchForETA & starting location & returns the ETA\footnotemark\  of the CAT\footnotemark\\ %
				\bottomrule
			\end{tabular}}\end{figure}
		\footnotetext{text}%TODO inserire le sezioni del RASD
		\footnotetext{text}
		
	\item \textit{Functionalities exploited by logged in user:} logged in user can access the searchForETA method exposed for guests along the following methods:
	\begin{figure}[H]	
		\makebox[\textwidth][c]{\begin{tabular}{*{3}{c}}
			\toprule
			Method & Parameters required & Notes \\
			\midrule
			requestTaxi &  & \\ %TODO finire
			cancelRequest & & \\ %TODO finire
			showRequestDetails & & \\
			reserveTaxi & &  \\
			cancelReservation & & \\
			showReservationDetails & & \\
			register & & \\
			login & & \\
			logout & & \\
			\bottomrule
		\end{tabular}}\end{figure}
	
\item \textit{Driver:}
	
	\begin{figure}[H]	
		\makebox[\textwidth][c]{\begin{tabular}{*{3}{c}}
		\toprule
		Method & Parameters required & Notes \\
		\midrule
		answerRequest & & \\ %TODO finire
		answerReservation & & \\ %TODO finire
		toggleState & & \\
		\bottomrule
	\end{tabular}}\end{figure}

%TODO dubbissimo: ma qui le interfacce sono semplicemente i 3 metodi che vengono messi a disposizione dell'admin per fare add update e delete degli account?
\begin{comment}

\item Admin
	
	\begin{tabular}{*{3}{c}}
		\toprule
		Method & Parameters required & Notes \\
		\midrule
		answerRequest & & \\ %TODO finire
		answerReservation & & \\ %TODO finire
		toggleState & & \\
		\bottomrule
	\end{tabular}
	\end{comment}	
	
	\paragraph{Notifications}
	This interface exposes methods to notify end users (except for administrators) of particular events. 
	
	
	
\end{itemize}
\subsection{Selected Architectural Styles and Patterns}
%client-server
%3 tier
%thin client
%DMZ
\subsection{Other Design Decisions}
%Phonegap
%Cloud
%


\section{User Interface Design}
This section has been explored in RASD's section 2.1.1 ``User Interfaces'' so we refer to that one.

\section{Requirements Traceability}
%Explain	 how the requirements you have defined in  the RASD map	 into the design	 elements that you have  defined in this document
%TODO move both funct and non funct and explain how are they respected

Functional requirements:
\begin{itemize}
	\item Data accessibility, modifiability and creation for the 4 kinds of users\footnotemark\ is granted thanks to the http interface exposed on the Internet.
	\item Notification delivery is granted through the interface exposed by the mobile applications and web applications.
\end{itemize}
Non functional requirements:
\begin{itemize}
	\item Secure channels will be established using SSL 
	\item Autentication will always be checked before accessing sensible data.
	\item Stability, scalability and availability of resources will be granted by using a cloud-based system. The deployment will not be made on phisical machines but virtual machines rented on reliable providers.
	%TODO add SSL to acronyms
\end{itemize}
\footnotetext{Guest, Administrator, Regeistered User and Taxi Driver (RASD section 1.6)}
\section{References}
\clearpage
\section{Appendix}
Appendix for Roberto Clapis\\
Work hours: 20
\begin{center}
	Software Used:\\
	\-\\
	\begin{tabular}{*{2}{c}}
		\toprule
		Task & Software \\
		\midrule
		Edit \LaTeX\ Source & Vim\\
		Edit Graphs Sources & Vim\\
		Edit sources for Sequence Diagrams & Vim\\
		Convert Sequence Diagrams to images & Quick Sequence Diagram Editor\\
		Generate and Raster directed graphs& Dot\\
		Generate and Raster undirected graphs& Fdp\\
		General images mangling and cropping & ImageMagick \& Shotwell\\
		Convert \LaTeX\ source to PDF & \LaTeX\-MK\\
		Spell Check & Aspell \\
		\LaTeX\ Check & LaCheck\\
		\bottomrule
	\end{tabular}
\end{center}
\-\\
\-\\
\begin{comment}
TODO Erica
Appendix for Erica Stella\\
Work hours: 40
\begin{center}
	Software Used:\\
	\-\\
	\begin{tabular}{*{2}{c}}
		\toprule
		Task & Software \\
		\midrule
		Edit \LaTeX\ Source & TexStudio\\
		Use Case Diagrams & Eclipse\\
		Class Diagram & Eclipse\\
		General images mangling and cropping & GIMP\\
		Mockup & Pencil\\
		\bottomrule
	\end{tabular}
\end{center}
\end{comment}
\end{document}
