\documentclass[10pt,xcolor={usenames,dvipsnames}]{beamer}
\graphicspath{{pics/}}
\fontfamily{pnc}
\usepackage{verbatim}
\usepackage{listings}

\title{Final Presentation}
\subtitle{MyTaxiService}
\author{Roberto Clapis, Erica Stella}
\institute{Politecnico di Milano}
\date{12/02/2016}
\subject{Software Engineering 2}

\usepackage{graphicx}
\usebackgroundtemplate{\includegraphics[width=\paperwidth,height=\paperheight]{background2.jpg}}

\setbeamertemplate{navigation symbols}{}
\setbeamercolor{frametitle}{fg=CornflowerBlue}
\setbeamercolor{title}{fg=CornflowerBlue}
\setbeamercolor{framesubtitle}{fg=RoyalBlue}

\addtobeamertemplate{frametitle}{\vskip+3ex}{} 

\begin{document}
\frame{\titlepage}
\begin{frame}
	\frametitle{Table of Contents}
	\tableofcontents[currentsection]
\end{frame}


\section[Section]{RASD}

\begin{frame}
	\begin{center}
		RASD	
	\end{center}
\end{frame}
%\begin{frame}
%	\frametitle{The RASD Document}
%	\framesubtitle{The approach}
%	We tried to solve all the easy problems with a simple approach, in order to focus on the actual difficulties. \\
%	\begin{itemize}
%		\item Login, logout, registration were kept as standard as possible
%		\item Usability was kept in mind, customizability was not considered as a feature
%		\item We tried to keep as little as possible the user inputs and let the application do the work
%	\end{itemize}
%\end{frame}
\begin{frame}
	\frametitle{The RASD Document}
	\framesubtitle{The goals}
	\begin{itemize}
		\item Provide an easy way to request a taxi.
		\item Provide an easy way to reserve a taxi.
		\item Guarantee a fair management of the taxi queues.
		\item Create an extensible system that allows expansion and interactions with other services.
	\end{itemize}
\end{frame}
 \begin{frame}
	\frametitle{Domain Assumptions}
	\framesubtitle{The main assumptions}
	\begin{itemize}
		\item All taxi drivers will have a phone with active GPS functionality 
		\item The GPS information about the position of the available taxi drivers sent to the system will always be accurate
		\item Taxi drivers' phones will always have an internet connection while myTaxiService is running
		\item Users will enter a valid email address during registration 
		\item Users who have requested or reserved a taxi will always be present when the taxi arrives.
	\end{itemize}
\end{frame}
\begin{frame}
	\frametitle{Requirements}
	\framesubtitle{The main functional requirements}
	For both the API and UI interfaces:
	\begin{itemize}
			\item The system will be able to calculate and show the ETA\@ 
			\item The system will provide a functionality to request a taxi at the given starting location 
			\item The system will provide a functionality to reserve a taxi at the given starting location and meeting time
			\item The system will provide a functionality to see currently active reservations and requests
			\item The system will provide a functionality to notify users of the code of the incoming taxi
	\end{itemize}
\end{frame}
\begin{frame}
	\frametitle{Requirements}
	\framesubtitle{Non functional requirements}
	\begin{itemize}
	\item \textit{Availability}
		\begin{itemize}
			\item The server of the application must always be available
			\item The app must never freeze 
			\item The system must store all of its data in an always-reachable database
			\item Regular backups will be made in order to reduce or prevent data loss
		\end{itemize}
	\item \textit{Security}
		\begin{itemize}
			\item In no situation sensible data will pass through an insecure channel
		\end{itemize}
\end{itemize}
\end{frame}
\begin{frame}
	\frametitle{Two simple interfaces}
	\framesubtitle{Adaptive web-based UI to ensure maintainability for both the web UI and mobile UI}
	\begin{center}
		\includegraphics[width=\textwidth,height=\textheight,keepaspectratio]{GuestInterface}
	\end{center}
\end{frame}
\begin{frame}
	\frametitle{Use Cases}
	\framesubtitle{All the actions allowed were planned and analysed}
	\begin{center}
		\includegraphics[width=\textwidth,height=\textheight,keepaspectratio]{UseCaseDiagram}
	\end{center}
\end{frame}
\begin{frame}
	\frametitle{Use Cases}
	\framesubtitle{Each use case is analysed in depth with a sequence diagram}
	\begin{center}
		\includegraphics[width=\textwidth,height=\textheight,keepaspectratio]{request-a-taxi}
	\end{center}
\end{frame}
\begin{frame}
	\frametitle{Class Diagram}
	\framesubtitle{The class diagram was kept as readable as possible, underlying the main decisions taken}
	\begin{center}
		\includegraphics[width=\textwidth,height=\textheight,keepaspectratio]{ClassDiagram}
	\end{center}
\end{frame}
\begin{frame}
	\frametitle{Alloy}
	\begin{center}
		\includegraphics[width=\textwidth,height=\textheight,keepaspectratio]{reserved-and-refused}
	\end{center}
\end{frame}

\section[Section]{Design Document}
\begin{frame}
	\begin{center}
		Design Document
	\end{center}
\end{frame}
\begin{frame}
	\frametitle{The Design}
	The architecture of the application is pretty standard
	\begin{itemize}
		\item Classical DMZ structure for the Network 
		\item Database + Web server for the components
	\end{itemize}
\end{frame}
\begin{frame}
	\frametitle{The Deployment}
	\framesubtitle{A standard DMZ + Internal network approach was used}
	\begin{center}
		\includegraphics[width=\textwidth,height=\textheight,keepaspectratio]{dot/deployment}
	\end{center}
\end{frame}
\begin{frame}
	\frametitle{The Component View}
	\begin{center}
		\includegraphics[width=\textwidth,height=\textheight,keepaspectratio]{dot/component}
	\end{center}
\end{frame}
\begin{frame}
	\frametitle{More on the Components}
	\framesubtitle{Extensibility and maintainability as targets}
	\begin{itemize}
		\item Both mobile and web version are based on web technologies.
		\item The queues, reservations and requests are computed by the DBMS.
		\item The web server ignores the fact that the client is using a mobile app or a browser.
		\item The APIs query the DBMS in the exact same way the Web Server does.
	\end{itemize}
\end{frame}
\begin{frame}
	\frametitle{More on the Components}
	\framesubtitle{Focus on the WebServer}
	\begin{center}
		\includegraphics[width=\textwidth,height=\textheight,keepaspectratio]{dot/Webserver}
	\end{center}

\end{frame}

\section[Section]{Test Document}
\begin{frame}
	\begin{center}
		Test Document	
	\end{center}
\end{frame}
\begin{frame}
	\frametitle{WebServer}
	\framesubtitle{}
		\includegraphics[width=\textwidth,height=\textheight,keepaspectratio]{WebServer1}\\
\end{frame}
 \begin{frame}
	\frametitle{WebServer}
	\framesubtitle{}
	\begin{center}
		\includegraphics[width=0.8\textwidth,height=0.8\textheight,keepaspectratio]{WebServer2}
	\end{center}
\end{frame}
\begin{frame}
	\frametitle{Webserver $\rightarrow$ UI}
	\framesubtitle{}
	\begin{center}
		\includegraphics[width=0.8\textwidth,height=0.8\textheight,keepaspectratio]{WSUI3}\\\-\\
		\includegraphics[width=0.8\textwidth,height=0.8\textheight,keepaspectratio]{WSUI1}\\\-\\
		\includegraphics[width=0.8\textwidth,height=0.8\textheight,keepaspectratio]{WSUI2}\\\-\\
	\end{center}
\end{frame}
\begin{frame}
	\frametitle{DBMS $\rightarrow$ Webserver}
	\framesubtitle{}
		\includegraphics[width=0.8\textwidth,height=0.8\textheight,keepaspectratio]{DBWS}
\end{frame}

%\begin{frame}
%	\frametitle{The Integration Sequence}
%	\framesubtitle{}
%	%TODO erica non ho capito cosa devo fare: ``ah poi nel testing io direi di far vedere l'ordine di integrazione''
%\end{frame}

\section[Section]{Project Plan}
\begin{frame}
	\begin{center}
		Project Plan
	\end{center}
\end{frame}
\begin{frame}
	\frametitle{Function Points}
	152 FP in Total:
	\begin{itemize}
		\item Internal Logical File 45 FP
		\item External Input 70 FP
		\item External Output 19 FP
		\item External Inquiry 18 FP
	\end{itemize}
\end{frame}
\begin{frame}
	\frametitle{COCOMO}
	\framesubtitle{}
	\begin{itemize}
		\item $152*46 \cong 7000$ estimated \textbf{S}ource \textbf{L}ines \textbf{O}f \textbf{C}ode
		\item $\sim 25$ Person-Month
		\item 3 People for a total of 10 Months
	\end{itemize}
\end{frame}
\begin{frame}
	\frametitle{Risks}
	\framesubtitle{}
	\begin{itemize}
		\item	Personnel shortfall due to any cause (e.g. illness)
		\item	Capacity shortfalls of the entire system
		\item	Problems with the mobile application development framework
		\item	Taxi drivers don't cooperate in the use of the application
		\item	The application is not well-accepted by the public
	\end{itemize}
\end{frame}

\section[Section]{Code Inspection}
\begin{frame}
	\begin{center}
		Code Inspection
	\end{center}
\end{frame}
\begin{frame}
	\frametitle{The Class}
	\framesubtitle{}
	The class was \texttt{ConnectorDeployer} from the package \texttt{com.sun.enterprise.connectors.module}
	The methods were
	\texttt{\begin{itemize}
		\item private void registerBeanValidator(String rarName, ReadableArchive archive, ClassLoader classLoader)
		\item private List \textless String\textgreater getValidationMappingDescriptors(ReadableArchive archive)
		\item public void event(Event event)
	\end{itemize}
	}
\end{frame}
\begin{frame}
	\frametitle{The Checklist}
	\framesubtitle{The Style is ok}
	\begin{itemize}
		\item Naming conventions were mainly respected
		\item Braces and indention respected the standard except for a couple of lines
		\item File organization and white spaces were correctly used. With the exception of a comment line that was too long and a couple of white lines to be removed.
		\item High level breaks were used in code lines
	\end{itemize}
\end{frame}
\begin{frame}
	\frametitle{The Checklist}
	\framesubtitle{But the style is not everything}
	\begin{itemize}
		\item	Documentation is mostly absent or incomplete\\
		\item  There is a condition of an if that was commented out with no reasons to explain why\\
		\includegraphics[width=0.8\textwidth,height=0.8\textheight,keepaspectratio]{commento}
		\item  One of the functions was definitely too complex, it represented the logical function of three different methods, one after the other, separated by a comment.\\
		%\item  An other couple should have been splitted in at least two simpler ones.
	\end{itemize}
\end{frame}
\begin{frame}
	\frametitle{The Checklist}
	\framesubtitle{A couple of exceptions}
	A couple of exceptions were caught but not handled, and both of them have a comment stating\\
	\-\\
	\texttt{{\color{Blue}try }\{\\\-	reader.close();\\\} {\color{Blue}catch} (Exception e) \{\\{\color{Green}\-	//ignore?}\\\}}
\end{frame}

\begin{frame}
	\frametitle{To close up}
	\framesubtitle{Closing is important}
	There is a method that opens a list of inputstreams from an archive file and if an exception is thrown the handler \underline{just closes the last inputstream} opened.\\
	\-\\
	According to the Java Documentation closing \underline{every} inputstream that refers to a file closes the file itself.\\
	%Even if a failure in that point of the application probably implies a failure of the whole process, that will soon close and free the file descriptors, this is a terrible practice and it should be avoided.
\end{frame}
\begin{frame}
	\frametitle{}
	\framesubtitle{}
	{\Huge
		\begin{center}
			That's all!
		\end{center}
	}
\end{frame}
\end{document}

