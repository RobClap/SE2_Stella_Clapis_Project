\documentclass{article}
\usepackage{adjustbox}
\usepackage{float}
\usepackage{textcomp}
\usepackage{graphicx}
\graphicspath{{images/}}
\usepackage{booktabs}
\usepackage{color}
\usepackage{verbatim}
\usepackage{listings}
\usepackage{underscore}
\setcounter{secnumdepth}{5}
\usepackage[bookmarks=true]{hyperref}
\author{Roberto Clapis (841859), Erica Stella (854443)} 
\date{\today}
\title{Politecnico di Milano
	\\A.A. 2015\@-\@2016
	\\Software Engineering 2: ``myTaxiService''
	\\\textbf{P}roject \textbf{P}lan}
\hypersetup{pdftitle={Project Plan},    % title
	pdfauthor={Roberto Clapis, Erica Stella},                     % author
	pdfsubject={Project Plan},                        % subject of the document
	pdfkeywords={TeX, LaTeX, taxi, PP, SoftwareEngineering2}, % list of keywords
	colorlinks=true,       % false: boxed links; true: colored links
	linkcolor=black,       % color of internal links
	citecolor=blue,       % color of links to bibliography
	filecolor=black,        % color of file links
	urlcolor=purple,        % color of external links
}
\begin{document}
\maketitle
\begin{center}
	\includegraphics{polimi-logo}
\end{center}
\clearpage
\tableofcontents
\clearpage
\section{Introduction}
This document describes the project plan for myTaxiService application.
It presents an analysis of the expected size and effort required 
for the implementation phase calculated respectively with the Function
Points and COCOMO. Then it presents the available resources and how 
they will be allocated to the project tasks and, in the end, it
discusses the possible risks this project might encounter and the
associated recovery actions.
\section{Size and Effort Estimation}
\subsection{Size Estimation - Function Points}
Following Albrecht's method, our application's function points
will be divided in 5 types:
\begin{itemize}
	\item Internal Logical File (ILF): homogeneous set of data used
	and managed by the application.
	\item External Interface File (EIF): homogeneous set of data 
	used by the application but generated and maintained by other.
	\item External Input: elementary operation to elaborate data
	coming from the external environment.
	\item External Output: elementary operation that generates data
	for the external environment.
	\item External Inquiry: elementary operation that involves
	input and output.
\end{itemize}
The function points' types stated above will be weighted as 
specified in the following table.
\begin{center}
	\includegraphics[width = 0.8\textwidth]{FP}
\end{center}
\subsubsection{Internal Logical File}
%TODO
user-username password hash email
taxi driver - taxi code username password hash email zona corrente - ora di attivazione
admin username password hash email
city zones with nearest zones
\subsubsection{External Interface File}
There are no such things in our project.
\subsubsection{External Input}
%TODO
registrazione
login
logout
request a taxi
reserve a taxi
modify personal data
switch taxi driver state
accept/refuse req and res
location of the taxi driver based on the gps
add a new taxi driver into the system
remove a taxi driver from the system
modify taxi drivers' credentials data
cancel req/res
\subsubsection{External Output}
%TODO
suggestion di completamento vie
notifica taxi code del taxi che sta arrivando
notify a request to the taxi driver
notify a reservation

calculate and show ETA - involves calculate the queues and eventually calculate the CAT
\subsubsection{External Inquiry}
%TODO
select a currently active req or res
select a taxi driver for the admin
show the list of taxi drivers for the admin
show all active req and res of a user
\subsection{Effort Estimation - COCOMO}
%TODO
\section{Resource Allocation}
%TODO descrizione
\subsection{Tasks}
RASD:
\begin{itemize}
	\item Domain assumptions
	\item Functional requirements
	\item Non functional requirements
	\item Use cases and scenarios
	\item User interface
	\item Alloy
\end{itemize}
Design Document:
\begin{itemize}
	\item
\end{itemize}
Code Inspection:
\begin{itemize}
	\item
\end{itemize}
Integration Test Plan Document:
\begin{itemize}
	\item 
\end{itemize}
\section{Risks}
\end{document}