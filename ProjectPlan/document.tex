\documentclass{article}
\usepackage{adjustbox}
\usepackage{float}
\usepackage{marvosym}
\usepackage{amsmath}
\usepackage{textcomp}
\usepackage{graphicx}
\graphicspath{{images/}}
\usepackage{booktabs}
\usepackage{color}
\usepackage{verbatim}
\usepackage{listings}
\usepackage{underscore}
\setcounter{secnumdepth}{5}
\usepackage[bookmarks=true]{hyperref}
\author{Roberto Clapis (841859), Erica Stella (854443)} 
\date{\today}
\title{Politecnico di Milano
	\\A.A. 2015\@-\@2016
	\\Software Engineering 2: ``myTaxiService''
	\\\textbf{P}roject \textbf{P}lan}
\hypersetup{pdftitle={Project Plan},    % title
	pdfauthor={Roberto Clapis, Erica Stella},                     % author
	pdfsubject={Project Plan},                        % subject of the document
	pdfkeywords={TeX, LaTeX, taxi, PP, SoftwareEngineering2}, % list of keywords
	colorlinks=true,       % false: boxed links; true: colored links
	linkcolor=black,       % color of internal links
	citecolor=blue,       % color of links to bibliography
	filecolor=black,        % color of file links
	urlcolor=purple,        % color of external links
}
\begin{document}
\maketitle
\begin{center}
	\includegraphics{polimi-logo}
\end{center}
\clearpage
\tableofcontents
\clearpage
\section{Introduction}
This document describes the project plan for myTaxiService application.
It presents an analysis of the expected size and effort required 
for the implementation phase, calculated respectively with the Function
Points and COCOMO\@. Then, it presents the available resources and how 
they will be allocated to the project tasks and, in the end, it
discusses the possible risks this project might encounter and the
associated recovery actions.
\section{Size and Effort Estimation}
\subsection{Size Estimation -- Function Points}
Following Albrecht's method, our application's function points
will be divided in 5 types:
\begin{itemize}
	\item Internal Logical File (ILF): homogeneous set of data used
	and managed by the application.
	\item External Interface File (EIF): homogeneous set of data 
	used by the application but generated and maintained by other.
	\item External Input: elementary operation to elaborate data
	coming from the external environment.
	\item External Output: elementary operation that generates data
	for the external environment.
	\item External Inquiry: elementary operation that involves
	input and output.
\end{itemize}
The function points' types stated above will be weighted as 
specified in the following table.
\begin{center}
\begin{table}[H]
\begin{tabular}{|*4{c|}}
\hline
Function Type&\multicolumn{3}{c|}{Complexity}\\
\cline{2-4}
&Simple&Medium&Complex\\
\hline
\hline
Internal Logic File & 7 & 10 & 15 \\
\hline
External Interface File & 5 & 7 & 10 \\
\hline
External Input & 3 & 4 & 6 \\
\hline
External Output & 4 & 5 & 7 \\
\hline
External Inquiry & 3 & 4 & 6 \\
\hline
\end{tabular}
%	\begin{tabular}{*{4}c}
%		\toprule
%		Function Type & \multicolumn{3}{|c|}{Complexity}\\
%		\midrule
%		Internal Logic File & 7 & 10 & 15 \\
%		\bottomrule
%	\end{tabular}
\end{table}
%	\includegraphics[width = 0.8\textwidth]{FP}
\end{center}
\subsubsection{Internal Logical File}
The application stores information about: 
\begin{itemize}
	\item Admins (user name, password hash, email)
	\item Users (user name, password hash, email)
	\item City zones (nearest zones)
	\item Requests (starting zone, destination zone, taxi code)
	\item Reservations (starting zone, destination zone, taxi code, meeting time)
	\item Taxi drivers (taxi code, user name, password hash, current zone, availability time)
\end{itemize}
The first five entities have a simple structure as they are composed 
of a small number of fields, while the taxi drivers have a more complex
structure that also needs to be updated frequently. Thus we decide to 
adopt the simple weight for the first five and the medium weight
for the last one.
5\texttimes7+1\texttimes10 = 45 FPs concerning ILFs.
\subsubsection{External Interface File}
There are no such things in our project.
\subsubsection{External Input}
myTaxiService application interacts with users
as follows: 
\begin{itemize}
	\item login/logout: as they're simple operations the simple weight can be adopted. 2\texttimes3 = 6 FPs.
	\item registration: this is considered of medium complexity as it requires a careful update of the ILFs. 1\texttimes4 = 4 FPs
	\item request/reserve a taxi: these operations are considered complex as they also involve the calculation of the queues and the CAT. 2\texttimes6 = 12 FPs
	\item cancel a request/reservation: these operation are complex as they involve two entities, users and taxi drivers, and a careful concurrency handling. 2\texttimes6 = 12 FPs
	\item modify personal data: this can be considered of medium complexity as it requires the handling of important information. 4 FPs
	\item toggle taxi driver state: this is considered an easy operation as it concerns only one entity. 3 FPs
	\item accept/refuse requests/reservations: these are considered operations of medium complexity as they require of two important entities. 4\texttimes4 = 16
	\item receiving GPS information on the taxi driver's location: this is considered a medium complexity operation. 4 FPs
\end{itemize}
It also interacts with admins as follows:
\begin{itemize}
	\item insert/modify/remove a taxi driver from the database: these can be considered easy operations. 3\texttimes3 = 9 FPs
\end{itemize}
Thus, 70 FPs concerning External Inputs.

\subsubsection{External Output}
The application allows the creation of:
\begin{itemize}
	\item notifications about the taxi code to send to users: we can adopt the simple weight for this operation. 4 FPs
	\item notifications of requests/reservations to send to taxi drivers: we can adopt the simple weight for these operations too. 2\texttimes4 = 8 FPs
	\item notifications of the ETA: this is a complex operation as it involves the calculation of the queues and of the CAT. 7 FPs
\end{itemize}
19 FPs concerning External Outputs.

\subsubsection{External Inquiry}
The application allows users to:
\begin{itemize}
	\item show all his/her active requests and reservations
	\item select a currently active request/reservation
	\item select the street between the available ones
\end{itemize}
and it allows admins to:
\begin{itemize}
	\item show the list of all taxi drivers
	\item select a taxi driver
\end{itemize}
These 6 operations can all be considered easy, so we can adopt the simple weight for all of them. 6\texttimes3 = 18 FPs concerning External Inquiries.

\subsection{Effort Estimation - COCOMO}
To convert the function points in SLOC we use a 
conversion factor of 46, as specified here 
http://www.qsm.com/resources/function-point-languages-table
for J2EE.
\begin{center}
	SLOC = FP \texttimes 46 = 152 \texttimes 46 = 6992
\end{center}
We consider our project with all nominal Cost Drivers and
Scale Drivers so we have an EAF of 1.00 and an
exponent E of 1.0997.
\begin{center}
	\EUR ffort = 2.94\texttimes EAF\texttimes $(KSLOC)^{E}$ = 2.94\texttimes$(6.992)^{1.0997}$ = 24.95 Person-Month
\end{center}
The duration of the project could therefore be calculated with the following equation
\begin{center}
	Duration = 3.67\texttimes(\EUR ffort)$^{SE}$ = 3.67\texttimes$24.95^{0.3179}$ = 10.2 months
\end{center}
where SE = 0.3179 as derived from nominal Scale Drivers.
Thus, the estimate of people needed to complete the project is
\begin{center}
	N$_{people}$ = $\frac{Effort}{Duration}$ = 2.45 people $\rightarrow$ 3 people
\end{center}

\section{Resource Allocation}
\subsection{Tasks}
We're going to refer to the tasks with a [Tx] notation, where x stands for a number used to uniquely identify the task. \\
RASD (15.10.2015-6.11.2015):
\begin{itemize}
	\item T1 - Domain assumptions
	\item T2 -  Functional requirements
	\item T3 -  Non functional requirements
	\item T4 - Use cases and scenarios
	\item T5 - Models (the world and the machine, class diagram, sequence diagrams)
	\item T6 - User interface
	\item T7 - Alloy
\end{itemize}
Design Document (12.11.2015-4.12.2015):
\begin{itemize}
	\item T8 - Component view
	\item T9 - Deployment view
	\item T10 - Runtime view
	\item T11 - Component interfaces
\end{itemize}
Code Inspection (9.12.2015-5.1.2016):
\begin{itemize}
	\item T12 - Checklist points 1-30
	\item T13 - Checklist points 31-60
\end{itemize}
Integration Test Plan Document (7.1.2016-21.1.2016):
\begin{itemize}
	\item T14 - Integration sequence
	\item T15 - Test case specifications
	\item T16 - Test procedures
	\item T17 - Tools, drivers and stubs
\end{itemize}
\begin{table}[H]
	\begin{tabular}{c c | c c | c c}
		\toprule
		Tasks & Dependencies &  Tasks & Dependencies & Tasks & Dependencies\\
		\midrule
		T1 & None&T7 & T2, T3&T13 & None\\
		T2 & T1&T8 & None&T14 & None\\
		T3 & T1&T9 & T8&T15 & T14\\
		T4 & T2, T3&T10 & T8&T16 & T15\\
		T5 & T4&T11 & T8&T17 & T15\\
		T6 & T4&T12 & None& & \\
		\bottomrule
	\end{tabular}
\end{table}
The following Gantt's graph describes by who the tasks stated above have been completed and when.\\
\begin{center}
\includegraphics[height = 1.0\textheight]{PROJECTPLAN}
\end{center}
\section{Risks}
\begin{table}[H]
	\begin{adjustbox}{width = 1.2\textwidth}
	\begin{tabular}{*{4}{p{0.6\textwidth}}}
		\toprule
		Risk & Probability & Effects & Strategy\\
		\midrule
		Personnel shortfall due to any cause (e.g. illness) & Moderate & Serious & allocate at least 2 people to each task so that if one goes missing the other one can carry on the job\\
		Capacity shortfalls of the entire system & High & Catastrophic & Use a real-time scalable cloud-based infrastructure  \\
		Problems with the mobile application development framework & Low & Serious & investigate several frameworks to better choose of the one to use\\
		Taxi drivers don't cooperate in the use of the application & Low & Catastrophic & Propose discounts and other advantages to those adhering to the application\\		
		The application is not well-accepted by the public & Medium & Catastrophic & Invest on the Marketing of the project \\
		\bottomrule
	\end{tabular}
	\end{adjustbox}
\end{table}
\end{document}
