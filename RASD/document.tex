\documentclass{article}
\usepackage{graphicx}
\graphicspath{{images/}}
\usepackage{booktabs}
\usepackage{verbatim}
\usepackage{listings}
\usepackage{underscore}
\setcounter{secnumdepth}{5}
\usepackage[bookmarks=true]{hyperref}
\author{Roberto Clapis (841859), Erica Stella (854443)} 
\date{\today}
\title{Politecnico di Milano
	\\A.A. 2015\@-\@2016
	\\Software Engineering 2: ``TAXInseconds''
	\\\textbf{R}equirements \textbf{A}nalysis \\and\\ \textbf{S}pecifications \textbf{D}ocument}
\hypersetup{pdftitle={Software Requirement Specification},    % title
	pdfauthor={Roberto Clapis, Erica Stella},                     % author
	pdfsubject={TeX and LaTeX},                        % subject of the document
	pdfkeywords={TeX, LaTeX, taxi, RASD, SoftwareEngineering2}, % list of keywords
	colorlinks=true,       % false: boxed links; true: colored links
	linkcolor=black,       % color of internal links
	citecolor=blue,       % color of links to bibliography
	filecolor=black,        % color of file links
	urlcolor=purple,        % color of external links
}
\begin{document}
\maketitle
\includegraphics{polimi-logo}
\clearpage
\tableofcontents
\clearpage

\section{Introduction}

\subsection{Purpose}
This document represent the Requirement Analysis and Specification Document (RASD). It aims at explaining the domain of the system to be developed and the system itself in terms of functional requirements, nonfunctional requirements and constraints. It also provides several models of the system and typical use cases. It is intended for all the developers who will have to implement the system, the testers who will have to determine if the requirements have been met and the system analysts who will have to write specifications for other system that will relate to this one. It is also intended as a contractual basis thus being legally binding.

\subsection{Actual System}
The government of the city wants to optimize its taxi service with a completely new application. Therefore, we assume there are no previous systems to take into account.


\subsection{Scope}
The aim of the project TAXINSECONDS is to provide a new application to optimize the taxi service of the city that will be accessible via browser, mobile or public APIs.
\\The application will be available to the users in web and mobile forms and will have public APIs in order to expand and improve the service with additional features.
\\The city managed by TAXINSECONDS is divided in zones of 2 km\textsuperscript{2} each and every zone has its own queue of taxis. The queues are automatically computed by the system with the information it receives from the GPS of the taxis. 
\\Taxi drivers can be available or not. Only available taxi drivers can be in a queue. When a taxi driver changes her state from not available to available the system automatically adds her to the queue of the zone she is currently in, based on the information of the GPS of her taxi. 
\\Users that are not registered can only see the estimated time of arrival of the nearest taxi with TAXINSECONDS.\@
\\Registered users can also request a taxi or make a reservation for a taxi. Reservations can only be made at least two hours before the ride and must be done specifying the starting location, the destination and the meeting time. Requests, instead, only need the starting location and the destination.
\\When a request or a reservation is made, the first taxi driver of the queue of the starting location's zone is notified for accepting or rejecting it. If a taxi driver rejects it her state is put on unavailable until she changes it back to available. If a taxi driver doesn't accept or reject within 1 minute, it will be passed on to the next taxi driver in the queue and the first one will be moved to the end of the queue. If there are no available taxis in the zone of the request the system will propagate the request to the closest available taxi.
\\When a request is accepted, the registered user that has made the request receives a notification from the system informing her of the code of the incoming taxi and the estimated time of arrival.
\\When a reservation is made, the system confirms it to the user and allocates a taxi 10 minutes before the meeting time. If a taxi for that zone is not available the request is then handled by finding the closest available taxi. When a taxi driver accepts the reservation, the user receives from the system the code of the incoming taxi. %TODO c'è da accettare le richieste o rifiutarle anche per le reservation-->fatte ma da controllare che vada bene %TODO non capisco perche differenziare le request dalle calls
\\Requests can be cancelled before they have been accepted by a taxi driver while reservation can be cancelled until 10 minutes before the meeting time.
\\%TODO se non si presentano entro 5 minuti pagano minimum fare

\subsection{Goals}
\begin{itemize}
	\item Provide an easy way to request a taxi.
	\item Provide an easy way to reserve a taxi.
	\item Guarantee a fair management of the taxis queue.
	\item Create an extensible system that allows expansion and interactions with other services.
\end{itemize}

\subsection{Definition and Acronyms}

\subsubsection{Definitions}
\begin{itemize}
	\item \textit{Guest:} a person that has to sign up or log in the system.
	\item \textit{Secure Channel:} a communication channel to ensure privacy and autenticity for both the server and the clients %TODO rende tutto molto più giusto e semplice,ho tolto la parte implementativa 
	\item \textit{Registered user:} a person that has already signed up and logged in the system. %TODO tremendamente imprecisa
	\item \textit{Administrator:} a person authorized to modify the list of taxi drivers stored by the system.
	\item \textit{Request:} a call from a registered user who needs a taxi immediately.
	\item \textit{Meeting time:} the date and time in which the registered user needs the taxi in case of reservation.
	\item \textit{Reservation:} a booking of a taxi at a certain meeting time.
	\item \textit{State of a taxi driver:} the state the taxi driver is currently in. It can be available or not available. Taxi drivers can be assigned to a request or a reservation by the system only if they're available.
	\item \textit{Closest available taxi:} if there are no taxis in the zone of the request or of the reservation, the system automatically finds the closest available taxi choosing the one with the smallest estimated time of arrival from the available taxis of the other zones.
\end{itemize}

\subsubsection{Acronyms}
\begin{itemize}
	\item ETA:\@ estimated time of arrival: the time, estimated by the system, that the closest available taxi will take to get to the starting location of the ride.
	\item CAT:\@ closest available taxi (see definition in the previous paragraph).
	\item API:\@ application programming interface is a set of routines, protocols, and tools for building software applications on top of this one.
\end{itemize}
\subsection{Actors}
\begin{itemize}
	\item \textit{Guest}: user are able to sign up, login or to ask the system for an ETA.\@
	\item \textit{Registered users}: after successfully logging in, registered users can request or reserve taxis or ask the system for an ETA.\@
	\item \textit{Taxi drivers}: after successfully logging in, taxi drivers are able to set their current state as available or not and to accept or refuse requests.
	\item \textit{Administrator}: after successfully logging in, the administrator will be the only user allowed to edit the taxi drivers list stored by the system.
\end{itemize}

\subsection{References}
\begin{itemize}
	\item The \href{run:./external_references/assignments.pdf}{document} with the assignment for the project
	\item The \href{run:./external_references/assignments.pdf}{IEEE Standard for SRS } 
\end{itemize}
\subsection{Overview}
This document is structured in three parts:
\begin{itemize}
	\item Introduction: gives an high-level description of the software purposes and context.
	\item Overall Description: gives a general description of the application, focusing on the context of the system, going in details about domain assumptions and constraintS. The aim of this section is to provide a context to the whole project and show its integration with the real world.
	\item Specific Requirements: this section contains all of the software requirements to a level of detail aimed to be enough to design a system to satisfy said requirements, and testers to test that the system actually satisfies them. It also contains the detailed description of the possible interactions between the system and the world with a simulation and preview of the expected response of the system with given stimulation. (Details are given with Alloy specifications and UML diagrams)
\end{itemize}


\section{Overall description}
\subsection{Product perspective}
The TAXINSECONDS application will be released as a web application and as a mobile application. 
There are no existing systems to integrate it with. 
\\It will provide a total of 4 main interfaces:
\begin{itemize}
	\item For both type of users
		\begin{itemize}
			\item Registered users
			\item Guests
		\end{itemize}
	\item For taxi drivers 
	\item For administrators
	\item A non graphical interface for APIs 
\end{itemize}

\subsubsection{User Interfaces}
%TODO Erica
\subsubsection{System interfaces}
\begin{comment}
	This should specify the following:
	a) The logical characteristics of each interface between the software product and its users. This
	includes those conÞguration characteristics (e.g., required screen formats, page or window layouts,
	content of any reports or menus, or availability of programmable function keys) necessary to accom-
	plish the software requirements.\
	b) All the aspects of optimizing the interface with the person who must use the system. This may simply
	comprise a list of doÕs and donÕts on how the system will appear to the user. One example may be a
	requirement for the option of long or short error messages. Like all others, these requirements
	should be veriÞable, e.g., Òa clerk typist grade 4 can do function X in Z min after 1 h of trainingÓ
	rather than Òa typist can do function X. Ó (This may also be speciÞed in the Software System
Attributes under a section titled Ease of Use.)
\end{comment}
\paragraph{Hardware Interfaces}
\begin{itemize}
\item Owned by the clients/drivers:
	\begin{itemize}
		\item Any GPS-capable device running Android>=4.0 or iOS>=6
		\item Any computer able to run a browser HTML5-compatible
	\end{itemize}
\item Owned by the company:
	\begin{itemize}
		\item The server on which the core of the application will run, and to which the applications, 
			the web UI and the API-related clients will connect to.
		\item Two machines optimized for a stateful firewall use
		\item A machine with the DBMS that will operate behind the second firewall 
	\end{itemize}
\end{itemize}
\paragraph{Software Interfaces}
\begin{itemize}
	\item Database Management System (DBMS):
		\begin{itemize}
			\item Name: MySQL.\@
			\item Version: 5.1.73 
			\item Source: https://www.mysql.com/
		\end{itemize}
	\item HTTP/HTTPS server:
		\begin{itemize}
			\item Name: Nginx
			\item Version: 1.8.0 
			\item Source: http://nginx.org/
		\end{itemize}
	\item PHP interpreter:
		\begin{itemize}
			\item Name: PHP
			\item Version: 5.3.3 
			\item Source: https://secure.php.net/
		\end{itemize}
	\item Operating System:
		\begin{itemize}
			\item Name: CentOS
			\item Version: 7.1--1503 
			\item Source: https://www.centos.org/ 
		\end{itemize}
\end{itemize}
\paragraph{Communication Interfaces}
\begin{center}
	\begin{tabular}{*{4}{c}}
	\toprule
    Protocol & Application layer Protocol & Port & Scope \\
	\midrule
	TCP & HTTP & 80 & Upgrade to a secure connection over HTTPS \\ 
    TCP & HTTPS & 443 & The web interface or the mobile apps \\ 
    TCP & HTTPS/JSON & 443 & The APIs \\ 
    TCP & HTTPS & 443 & The web interface or the mobile apps \\ 
    TCP & DBMS over SSL & 3306 & Communication between the webserver and the DBMS \\ 
    \bottomrule
    \end{tabular}
\end{center}
\paragraph{Memory constraints} 
\begin{itemize}
	\item Primary memory:
		\begin{itemize}
			\item for both taxi dirvers' and clients's mobile devices at least 500MB 
			\item for the web application 1GB or more is suggested
			\item for the server it is suggested to use a cloud service in order to resize memory according to  traffic
		\end{itemize}
	\item Secondary memory:
		\begin{itemize}
			\item mobile devices will need to have 50MB of free space on the device
			\item the web application require little to none secondary memory
			\item for the server it is suggested to use a cloud service in order to resize memory according to traffic
		\end{itemize}
\end{itemize}
\paragraph{Operations} 
%TODO
\begin{comment}
This should specify the normal and special operations required by the user such as
The various modes of operations in the user organization (e.g., user-initiated operations);
Periods of interactive operations and periods of unattended operations;
Data processing support functions;
Backup and recovery operations.
\end{comment}
\subsection{Product functions}
\subsection{User characteristics}
\subsection{Constraints}
\subsection{Assumptions and dependencies}
\subsubsection{Domain Assumptions}
\begin{itemize}
	\item All taxi drivers who intend to use the service will have a mobile phone with one of the supported mobile OSs
	\item All taxi drivers will have a phone with GPS functionality
	\item The taxi drivers will have to grant the system the rights to handle their taxi codes
	\item Taxi drivers' phones will have an internet connection
	\item Clients will have access to the internet and a device with GPS integrated 
	\item Clients have a valid email address
	\item Clients have a credit/debt card 
	\item Clients allow the app to access their credit in order to pay for the service
	\item When accessing from the website the clients will have to grant permission to the application to read the device's position
	\item The Owner of the app will have to build or rent an always-on DBMS and host for the Server-Side part of the app
%too obvious	\item A taxi driver can't be available and not available at the same time
\end{itemize}
\clearpage
\section{Specific requirements}
\subsection{Functional Requirements}
On the user side:\@
\begin{itemize}
	\item For the non logged in user (on both the WEB and Mobile interface):
		\begin{itemize}
			\item Provide information about taxis queues and availability 
			\item Registration in order to become a registered user
			\item Login for registered users
			\item ``change personal data'' procedure for registered users
		\end{itemize}
	\item For the logged in user (on both the WEB and Mobile interface):
		\begin{itemize}
			\item Provide information about taxis queues and availability 
			\item Place a call for a taxi at the current position 
			\item Make a reservation for later
			\item Logout
		\end{itemize}
	\item Through the API:\@
		\begin{itemize}
			\item Get the queue status for a given position
			\item Get the ETA for a taxi for a given position
			\item Establish a Secure Channel 
			\item Using a Secure Channel and valid credentials:\@
				\begin{itemize}
					\item Place a call for a given position
					\item Place a reservation for a given position
					\item Require the system to send a push message for updates about the status of a previous request
				\end{itemize}
		\end{itemize}
\end{itemize}
On the taxi driver side (there will be no Web interface in this case):\@
\begin{itemize}
	\item For the non logged in driver:
		\begin{itemize}
			\item Login for a registered driver 
			\item ``Reset password'' procedure for registered drivers
		\end{itemize}
	\item For the logged in driver:
		\begin{itemize}
			\item If credentials have been invalidated a ``reset password'' procedure will be mandatory
			\item Toggle the availability of the driver
			\item Notify for new calls
			\item Allow to accept or refuse a call
			\item Handle disconnection 
			\item Handle timeout (if reaching the client takes too much time)
			\item Log out
		\end{itemize}
	\item Through the API:\@
		\begin{itemize}
			\item Establish a Secure Channel
			\item Through a Secure Channel and with valid credentials:\@
				\begin{itemize}
					\item Toggle the driver state
					\item Send a notification when a call is made for the driver
					\item Accept the answer to the call, whether the answer is Acceptance or Refusal
					\item Handle disconnection
					\item Handle timeout (if no taxi reach the client in time)%TODO e come fa a saperlo?
				\end{itemize}
			\item 
		\end{itemize}
\end{itemize}
On the admin side:\@
\begin{itemize}
	\item For the non logged in admin:
		\begin{itemize}
			\item Log in
		\end{itemize}
	\item For the logged in admin:
		\begin{itemize}
			\item Add a new taxi driver
			\item Remove a taxi driver
			\item Invalidate a taxi driver's credentials 
			\item Set the area of the map handled by the system
			\item Change credentials for the admin
			\item Log out
		\end{itemize}
\end{itemize}
\subsection{Non functional requirements}
\begin{itemize}
	\item \textit{Cross platform}
		\begin{itemize}
			\item There will be a UI for mobile and one for the Web platform
			\item At least Android>=4.0 and iOS will be supported for the mobile version %TODO insert a meaningful iOS version
			\item At least Edge, Chrome and Firefox will ahve to be supported
			\item The web application will have to use HTML5
		\end{itemize}
	\item \textit{Availability}
		\begin{itemize}
			\item The application must always respond, even if no taxi are available at least with a ``No taxi available'' response
			\item If a query is taking some time a spinner will be shown, the app must never freeze
			\item The system must store data for both drivers and clients in an always-reachable database
			\item Regular Backups will be made in order to reduce or prevent data loss
		\end{itemize}
	\item \textit{Usability}
		\begin{itemize}
			\item The UIs have to be user friendly and with a few, clear functionalities
			\item The app will have few activities, mainly the login, the register and the call form.
			\item When inserting a location for a reservation the user will be helped with autocompletion for locations.
		\end{itemize}
	\item \textit{Security}
		\begin{itemize}
			\item In no situation sensible data will pass through an insecure channel
			\item No one should generally be able to impersonate the taxi drivers, the clients or the admin without proper autentication
		\end{itemize}
%	\item 
\end{itemize}
\subsection{Scenarios}
\subsubsection{Scenario 1}
Blair The Witch had to take her magical broom to the mechanic for the annual revision but she needs to go shopping to refill her stockpile of frog?s tails. Her friend Mizune has told her about TAXInseconds, so she decides to give it a try. After downloading it on her smartphone, she signs up compiling the registration form with her username, password, email and credit card data. Now she can complain about how slow car-based transports are!
\subsubsection{Scenario 2}
Suzuka is having a date tonight but, unfortunately, her car doesn?t want to start. After several failures, she decides to use TAXInasecond. After logging in, she requests a taxi specifying her home as the starting location and the restaurant?s address as the destination. The system notifies Takeshi, the first taxi driver of the queue in Suzuka?s zone, of the request. Takeshi accepts the request and the system sends Suzuka the code of Takeshi?s taxi and the ETA so she can finally get to her date.
\subsubsection{Scenario 3}
Ash Ketchum has to start his new adventure in Hoenn tomorrow but his mom is busy cleaning the house with Mr.\ Mime so she can?t take him to the airport to meet Prof.\ Oak. Ash decides to use once again TAXINSECONDS to reserve a taxi. After logging in, he makes a reservation for a taxi specifying the starting location as Pallet town, the destination and the meeting time. The morning after, the system allocates Brock?s taxi for the reservation and sends his code to Ash so that he knows who he?ll meet to start his new adventure. 
\end{document}


%
%	\begin{itemize}
%		\item  
%	\end{itemize}
