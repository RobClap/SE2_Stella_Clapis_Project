\documentclass{article}
\usepackage{graphicx}
\graphicspath{{images/}}
\usepackage{booktabs}
\usepackage{color}
\usepackage{verbatim}
\usepackage{listings}
\usepackage{underscore}
\setcounter{secnumdepth}{5}
\usepackage[bookmarks=true]{hyperref}
\author{Roberto Clapis (841859), Erica Stella (854443)} 
\date{\today}
\title{Politecnico di Milano
	\\A.A. 2015\@-\@2016
	\\Software Engineering 2: ``TAXInseconds''
	\\\textbf{R}equirements \textbf{A}nalysis \\and\\ \textbf{S}pecifications \textbf{D}ocument}
	\hypersetup{pdftitle={Software Requirement Specification},    % title
	pdfauthor={Roberto Clapis, Erica Stella},                     % author
	pdfsubject={TeX and LaTeX},                        % subject of the document
	pdfkeywords={TeX, LaTeX, taxi, RASD, SoftwareEngineering2}, % list of keywords
	colorlinks=true,       % false: boxed links; true: colored links
	linkcolor=black,       % color of internal links
	citecolor=blue,       % color of links to bibliography
	filecolor=black,        % color of file links
	urlcolor=purple,        % color of external links
}
\begin{document}
\maketitle
\includegraphics{polimi-logo}
\clearpage
\tableofcontents
\clearpage

\section{Introduction}

\subsection{Purpose}
This document represent the Requirement Analysis and Specification Document (RASD). It aims at explaining the domain of the system to be developed and the system itself in terms of functional requirements, nonfunctional requirements and constraints. It also provides several models of the system and typical use cases. It is intended for all the developers who will have to implement the system, the testers who will have to determine if the requirements have been met and the system analysts who will have to write specifications for other system that will relate to this one. It is also intended as a contractual basis thus being legally binding.

\subsection{Actual System}
The government of the city wants to optimise its taxi service with a completely new application. Therefore, we assume there are no previous systems to take into account.

\subsection{Scope}
The aim of the project TAXInseconds is to provide a new application to optimise the taxi service of the city that will be accessible via browser, mobile or public APIs.
\\The city managed by TAXInseconds is divided in zones of 2 km\textsuperscript{2} each and every zone has its own queue of taxis. The queues are automatically computed by the system with the information it receives from the GPS of the phones of the taxi drivers.
\\Taxi drivers can be available or not. Only available taxi drivers can be in a queue. When a taxi driver changes her state from not available to available the system automatically adds her to the queue of the zone she is currently in, based on the information of the GPS of her phone. 
\\Users that are not registered can only see the estimated time of arrival of the nearest taxi with TAXInseconds.\@
\\Registered users can also request a taxi or make a reservation for a taxi. Reservations can only be made at least two hours before the ride and must be done specifying the starting location, the destination and the meeting time. Requests, instead, only need the starting location and the destination.
\\When a request is made, the first taxi driver of the queue of the starting location's zone is prompted to accept or reject it. If the taxi driver rejects it her state is automatically put on unavailable by the system. If a taxi driver doesn't accept or reject the request within 1 minute, it will be passed on to the next taxi driver in the queue and the first one will be moved to the end of the queue. If there are no available taxis in the zone of the request the system will propagate the request to the closest available taxi.
\\When a request is accepted, the user that has made the request receives a notification from the system informing her of the code of the incoming taxi and the estimated time of arrival.
\\When a reservation is made, the system confirms it to the user and allocates a taxi 10 minutes before the meeting time. If a taxi for that zone is not available the closest available taxi will be notified. When a taxi driver accepts the reservation, the user receives from the system the code of the incoming taxi. If a taxi driver doesn't accept or reject the reservation within 1 minute, it will be passed on to the next taxi driver in the queue and the first one will be moved to the end of the queue.
\\Requests can be cancelled before they have been accepted by a taxi driver while reservation can be cancelled until 10 minutes before the meeting time.

\subsection{Goals}
\begin{itemize}
	\item Provide an easy way to request a taxi.
	\item Provide an easy way to reserve a taxi.
	\item Guarantee a fair management of the taxi queues.
	\item Create an extensible system that allows expansion and interactions with other services.
\end{itemize}

\subsection{Definition and Acronyms}

\subsubsection{Definitions}
\begin{itemize}
	\item \textit{Guest:} a person that has to sign up or log in the system.
	\item \textit{Secure Channel:} a communication channel to ensure privacy and authenticity for both the server and the clients 
	\item \textit{Logged in user:} a person that has already signed up and logged in the system.
	\item \textit{Administrator:} a person authorised to modify the list of taxi drivers stored by the system.
	\item \textit{Request:} a call from a registered user who needs a taxi immediately.
	\item \textit{Meeting time:} the date and time in which the registered user needs the taxi in case of reservation. %TODO ricordare che meeting time comprende anche la data -> erica polla
	\item \textit{Reservation:} a booking of a taxi at a certain meeting time.
	\item \textit{State of a taxi driver:} the state the taxi driver is currently in. It can be available or not available. Taxi drivers are in a queue if and only if they're available.
	\item \textit{Closest available taxi:} if there are no taxis in the zone of the request or of the reservation, the system automatically finds the closest available taxi choosing the one with the smallest estimated time of arrival from the taxi queues of the other zones.
\end{itemize}

\subsubsection{Acronyms}
\begin{itemize}
	\item ETA:\@ estimated time of arrival: the time, estimated by the system, that the closest available taxi will take to get to the starting location of the ride.
	\item CAT:\@ closest available taxi (see definition in the previous period).
	\item API:\@ application programming interface; it is a set of routines, protocols, and tools for building software applications on top of this one.
	\item MAD:\@ maximum allowed delay; the maximum time, calculated by the system according to its information about distance and traffic, that a taxi driver has to get to the starting location of a request. 
\end{itemize}
\subsection{Actors}
\begin{itemize}
	\item \textit{Guest}: guests are able to sign up, login or ask the system for an ETA.\@
	\item \textit{Logged in users}: after successfully logging in, registered users can request or reserve taxis or ask the system for an ETA.\@
	\item \textit{Taxi drivers}: after successfully logging in, taxi drivers are able to set their current state as available or not and to accept or refuse requests.
	\item \textit{Administrator}: after successfully logging in, the administrator will be the only user allowed to edit the taxi drivers list stored by the system.
\end{itemize}

\subsection{References}
\begin{itemize}
	\item The \href{run:./external_references/assignments.pdf}{document} with the assignment for the project
	\item The \href{run:./external_references/assignments.pdf}{IEEE Standard for SRS } 
\end{itemize}
\subsection{Overview}
This document is structured in three parts:
\begin{itemize}
	\item Introduction: gives an high-level description of the software purposes and context.
	\item Overall Description: gives a general description of the application, focusing on the context of the system, going in details about domain assumptions and constraints. The aim of this section is to provide a context to the whole project and show its integration with the real world.
	\item Specific Requirements: this section contains all of the software requirements to a level of detail aimed to be enough to design a system to satisfy said requirements, and testers to test that the system actually satisfies them. It also contains the detailed description of the possible interactions between the system and the world with a simulation and preview of the expected response of the system with given stimulation. (Details are given with Alloy specifications and UML diagrams)
\end{itemize}


\section{Overall description}
\subsection{Product perspective}
The TAXInseconds application will be released as a web application and as a mobile application. 
There are no existing systems to integrate it with. 
\\It will provide a total of 4 main interfaces:
\begin{itemize}
	\item For both type of users
		\begin{itemize}
			\item Registered users
			\item Guests
		\end{itemize}
	\item For taxi drivers 
	\item For administrators
	\item A non graphical interface for APIs 
\end{itemize}

\subsubsection{User Interfaces}
This section presents some mockups to provide an approximate idea of how the application's pages will be structured.
\clearpage
\paragraph{Guest interface}
This is the first page of the application where guests can choose to register, login or see the ETA.\\
\begin{center}
	\includegraphics[width=.9\textwidth,height=.9\textheight,keepaspectratio]{GuestInterface}
\end{center}
\clearpage
\paragraph{Login}
This is the login page where guests can log in the application and become either logged in users, administrators or taxi drivers.\\
\begin{center}
	\includegraphics[width=.9\textwidth,height=.9\textheight,keepaspectratio]{LoginInterface}
\end{center}
\clearpage
\paragraph{Register}
This is the mockup of registration form.\\
\begin{center}
	\includegraphics[width=.9\textwidth,height=.9\textheight,keepaspectratio]{RegisterInterface}
\end{center}
\clearpage
\paragraph{ETA}
This is the page where guests or logged in users can see the ETA to their starting location.\\
\begin{center}
	\includegraphics[width=.9\textwidth,height=.9\textheight,keepaspectratio]{ETA-interface}
\end{center}
\clearpage
\paragraph{Logged in user interface}
This is the homepage of a logged in user where she can also see all the active requests and reservations.\\
\begin{center}
	\includegraphics[width=.9\textwidth,height=.9\textheight,keepaspectratio]{LoggedinUserInterface}
\end{center}
\clearpage
\paragraph{Request a taxi interface}
This is the mockup of the page where a logged in user can request a taxi\\
\begin{center}
	\includegraphics[width=.9\textwidth,height=.9\textheight,keepaspectratio]{RequestTaxiInterface}
\end{center}
\clearpage
\paragraph{Reserve a taxi interface}
This is the mockup of the page where a logged in user can reserve a taxi\\
\begin{center}
	\includegraphics[width=.9\textwidth,height=.9\textheight,keepaspectratio]{ReserveTaxi}
\end{center}
\clearpage
\paragraph{Administrator interface}
This is the page of an administrator for which there is no mobile\\ 
\begin{center}
	\includegraphics[width=.8\textwidth,height=.8\textheight,keepaspectratio]{ReserveTaxi}
\end{center}
>>>>>>> origin/master
\clearpage
\subsubsection{System interfaces}
\begin{comment}
This should list each system interface and identify the functionality of the software to accomplish the system requirement and the interface description to match the system.
\end{comment} %TODO non ho capito davvero cosa sia questa %non dovrebbe essere semplicemente il titolo che raccoglie el successive? in teoria no, è una subsection come le altre %TODO ottimo, e che ci facciamo? secondo me è da togliere anche perchè non saprei cosa metterci
\subsubsection{Hardware Interfaces}
\begin{itemize}
	\item Owned by the users and taxi drivers:
		\begin{itemize}
			\item Any device running Android 4.0+ or iOS 6+ (GPS capability will make more functionalities available)
			\item Any computer able to run an HTML5-compatible browser 
		\end{itemize}
	\item Owned by the company:
		\begin{itemize}
			\item The server on which the core of the application will run, and to which the applications, 
				the web UI and the API-related clients will connect to.
%		\item Two machines optimised for a stateful firewall use
			\item A machine with the DBMS 
		\end{itemize}
\end{itemize}
\paragraph{Software Interfaces}
\begin{itemize}
	\item Database Management System (DBMS):
		\begin{itemize}
			\item Name: MySQL.\@
			\item Version: 5.1.73 
			\item Source: https://www.mysql.com/
		\end{itemize}
	\item HTTP/HTTPS server:
		\begin{itemize}
			\item Name: Nginx
			\item Version: 1.8.0 
			\item Source: http://nginx.org/
		\end{itemize}
	\item PHP interpreter:
		\begin{itemize}
			\item Name: PHP
			\item Version: 5.3.3 
			\item Source: https://secure.php.net/
		\end{itemize}
	\item Operating System: 
		\begin{itemize}
			\item Name: CentOS
			\item Version: 7.1--1503 
			\item Source: https://www.centos.org/ 
		\end{itemize}
\end{itemize}
\paragraph{Communication Interfaces}
\begin{center}
	\begin{tabular}{*{4}{c}}
		\toprule
		Protocol & Application layer Protocol & Port & Scope \\
		\midrule
		TCP & HTTP & 80 & Upgrade to a secure connection over HTTPS \\ 
		TCP & HTTPS & 443 & The web interface or the mobile apps \\ 
		TCP & HTTPS/JSON & 443 & The APIs \\ 
		TCP & HTTPS & 443 & The web interface or the mobile apps \\ 
		TCP & DBMS over SSL & 3306 & Communication between the webserver and the DBMS \\ 
		\bottomrule
	\end{tabular}
\end{center}
\paragraph{Memory constraints} 
\begin{itemize}
	\item Primary memory:
		\begin{itemize}
			\item for both taxi drivers' and clients' mobile devices at least 500MB 
			\item for the web application 1GB or more is suggested
			\item for the server it is suggested to use a cloud service in order to resize memory according to traffic
		\end{itemize}
	\item Secondary memory:
		\begin{itemize}
			\item mobile devices will need to have 50MB of free space on the device
			\item the web application requires no secondary memory 
			\item for the server it is suggested to use a cloud service in order to resize memory according to traffic
		\end{itemize}
\end{itemize}
\subsection{User characteristics}
The TAXInseconds application is intended for all users who are at least 18 years old. %TODO se non serve più la carta di credito possiamo estenderla ad altri user
\subsection{Assumptions and dependencies}
\subsubsection{Domain Assumptions}
\begin{itemize}
	\item All taxi drivers who intend to use the service will have a mobile phone with one of the supported mobile OSs
	\item All taxi drivers will have a phone with active GPS functionality 
	\item The taxi drivers will grant the system the rights to handle their taxi codes
	\item Taxi drivers' phones will always have an internet connection while TAXInseconds is running
	\item Users will have access to the internet
	\item In order to help the client to select the current location a phone with GPS capability will be required %TODO non credo sia una domain assumption %cos'è allora? credo sia da togliere tanto la starting location la mettono a mano
	\item Users will enter a valid email address during registration 
	\item Users will enter valid credit card data during registration
	\item Users allow the app to access their credit in order to pay for the service
	\item The Owner of the app will have to build or rent an always-on DBMS and host for the Server-Side part of the app %per ora lascialo 
	\item Users who have requested or reserved a taxi will always be present when the taxi arrives.
	\item There is always at least an available taxi to fulfil a request or a reservation in the whole city.
\end{itemize}
\clearpage
\section{Specific requirements}
\subsection{Functional Requirements}
On the user side:\@
\begin{itemize}
	\item For non logged in users (on both the WEB and Mobile interface):
		\begin{itemize}
			\item The system will be able to calculate and show the ETA.\@ 
			\item The system will provide a registration functionality 
			\item The system will provide a login functionality
		\end{itemize}
	\item For logged in users (on both the WEB and Mobile interface):
		\begin{itemize}
			\item The system will be able to calculate and show the ETA.\@ 
%TODO questa è una nonfunctional requirement %no questo è un funct req :3			\item The system will store the username, password, email and credit card data of every user %TODO non trovo più dove c'erano scritte le altre cose che devi scrivere nel login e che deve salvare-->sotto ho scritto magari di metterle in un posto comune con altre funct
			\item The system will provide a functionality to request a taxi at the given starting location 
			\item The system will provide a functionality to reserve a taxi at the given starting location and meeting time
			\item The system will provide a functionality to modify the personal data of the user
			\item The system will provide a logout functionality
			\item The system will provide a functionality to notify users of the code of the incoming taxi
			\item The system will provide a functionality to notify users of the ETA of the incoming taxi
			\item The system will provide a logout functionality
		\end{itemize}
	\item Through the API:\@ %TODO c'è qualquadra che non cosa, ti va se ne parliamo quando hai tempo?
		\begin{itemize}
			\item The system will be able to calculate and show the ETA.\@ 
			\item Establish a Secure Channel 
			\item Using a Secure Channel and valid credentials:\@
				\begin{itemize}
					\item Place a request for a given starting location
					\item Place a reservation for given starting location and meeting time
					\item Require the system to send a push message for updates about the status of a previous request
				\end{itemize}
		\end{itemize}
\end{itemize}
On the taxi driver side:\@ 
\begin{itemize}
	\item If credentials have been invalidated a ``reset password'' procedure will be mandatory %TODO ?.? ti spiego luned\label{}<++>
	\item The system will provide a functionality to switch the state of the taxi driver from available to not available or vice versa
	\item The system will notify the taxi driver of an incoming request showing the starting location
	\item The system will notify the taxi driver of a reservation showing the starting location and the meeting time
	\item The system will provide a functionality to allow taxi drivers to accept or reject reservations and requests
	\item Handle timeout (if reaching the client takes too much time) %TODO questa funzione e le altre per gestire le eccezioni secondo me sono da spostare in una sottosezione utilities insieme ai pagamenti, al fatto che controlla la posizione dei taxi driver, al computare le queue e metterei cosa salva di ogni utente %TODO non ho capito un cazzo
	\item The system will provide a logout functionality
\end{itemize}
Through the API:\@ %TODO le api mi sono oscure T.T %TODO te le spiego lunedi
\begin{itemize}
	\item Establish a Secure Channel
	\item Through a Secure Channel and with valid credentials:\@
		\begin{itemize}
			\item Toggle the driver state
			\item Send a notification when a call is made for the driver
			\item Accept the answer to the call, whether the answer is Acceptance or Refusal
			\item Handle disconnection
			\item Handle timeout (if no taxi reach the client in time)
		\end{itemize}
	\item 
\end{itemize}
On the admin side:\@
\begin{itemize}
	\item Add a new taxi driver
	\item Remove a taxi driver
	\item Invalidate a taxi driver's credentials %TODO se rimuove il taxista dal db non può più accedere per cui non deve invalidare le sue credenziali-->dobbiamo scrivere da qualche parte che l'accesso viene fatto su credenziali? %TODO invalidarle, non eliminarlo, significa che sospettano che qualcuno le abbia rubate, quindi gliele annullano, e lui deve selezionare una nuova pw
	\item Set the area of the map handled by the system %TODO davvero? tra l'altro, dove mettiamo le cose che gli input sono suggeriti? nelle funct degli utenti? %TODO vedi ``USABILITY'' in nonfunct req. (non sono cosi' ovvie a quanto pare)
	\item Modify registration data
	\item Log out
\end{itemize}
\subsection{Non functional requirements}
\begin{itemize}
	\item \textit{Cross platform} %TODO sono un po in dubbio su questo req %TODO ovvero?
		\begin{itemize}
			\item There will be a UI for mobile and one for the Web platform 
			\item At least Android 4.0+ and iOS 6+ will be supported for the mobile version 
			\item At least Edge, Chrome and Firefox will have to be supported
			\item The web application will have to use HTML5 %TODO questo non credo sia un nonfunctional req %TODO allora cosa è?
		\end{itemize}
	\item \textit{Availability}
		\begin{itemize}
			\item The application must always be available
			\item If a query is taking some time a spinner will be shown, the app must never freeze %TODO questo non è un req di availability %TODO quindi cosa è?
			\item The system must store all of its data in an always-reachable database
			\item Regular backups will be made in order to reduce or prevent data loss
		\end{itemize}
	\item \textit{Usability} %TODO credo che usability non sia un req, nel senso che da noi han detto che è una di quelle cose che dici "ma va?" %TODO ma se sopra mi hai detto di mettere che l'utente è aiutato con posizioni suggerite..%TODO si ma lo pensavo più come una qualche funzione 
		\begin{itemize}
			\item The UIs have to be user friendly and with a few, clear functionalities
			\item The app will have few activities, mainly the login, the register and the call form.
			\item When inserting a location for a reservation the user will be helped with auto completion for locations.
		\end{itemize}
	\item \textit{Security}
		\begin{itemize}
			\item In no situation sensible data will pass through an insecure channel
			\item No one should be able to impersonate the taxi drivers, the clients or the admin without proper authentication %TODO mmmhhh questa non mi convince %TODO preferisci che sia vero il contrario?
		\end{itemize}
%	\item 
\end{itemize}
\subsection{Scenarios}
\subsubsection{Scenario 1}
Blair The Witch had to take her magical broom to the mechanic for the annual revision but she needs to go shopping to refill her stockpile of frog's tails. Her friend Mizune has told her about TAXInseconds, so she decides to give it a try. After downloading it on her smartphone, she signs up compiling the registration form with her username, password and email. Now she can complain about how slow car-based transports are!
\subsubsection{Scenario 2}
Suzuka is having a date tonight but, unfortunately, her car doesn't want to start. After several failures, she decides to use TAXInseconds\@. After logging in, she requests a taxi specifying her home as the starting location and the restaurant's address as the destination. The system notifies Takeshi, the first taxi driver of the queue in Suzuka's zone, of the request. Takeshi accepts the request and the system sends Suzuka the code of Takeshi's taxi and the ETA so she can finally get to her date.
\subsubsection{Scenario 3}
Ash Ketchum has to start his new adventure in Hoenn tomorrow but his mother is busy cleaning the house with Mr.\ Mime, so she can't take him to the airport to meet Prof.\ Oak. Ash decides to use once again TAXInseconds to reserve a taxi. After logging in, he makes a reservation for a taxi specifying the starting location as Pallet town, the destination and the meeting time. The morning after, 10 minutes before the meeting time, the system allocates Brock's taxi for the reservation and sends his code to Ash so that he knows who he'll meet to start his new adventure. 
\subsubsection{Scenario 4}
Donald Duck is ready to start his first day as a taxi driver in Paperopoli. He jumps on his car, logs in TAXInseconds and changes his state to available. Unfortunately he doesn't know that, in his zone, there are 15 taxis in the queue before him, so he'll have to be patient for some time.
\subsubsection{Scenario 5}
After quite some time, Daisy requests a taxi and it's finally Donald's turn! The system notifies him but\ldots it's breakfast time and Donald's having a cappuccino in a very crowded bar. He doesn't hear the notification popping on his phone so, after 1 minute, the system forwards Daisy's request to the next taxi driver in the queue and changes Donald's state to not available.
\subsubsection{Scenario 6}
Mickey Mouse is planning to go fishing with Pluto tomorrow. After logging in TAXINSECONDS, he makes two reservations: one for the morning and one for the evening. But, while he's enjoying his evening watching a movie, the phone rings. It's Minnie and she's reminding Mickey that they he had promised to go shopping with her the day after. Sadly, he has to say goodbye to his fishing trip and cancels the taxi reservations.

\subsection{Use cases}
\subsubsection{Request a taxi}
\textit{Actors:} Logged in user, taxi driver
\textit{Preconditions:} The user is on the homepage of the application
\textit{Flow of Events:}
\begin{itemize}
	\item  The user clicks the ``Request a taxi'' button
	\item  The system shows the user a page where she can enter the starting location and destination of the request
	\item  The user inserts the starting location and the destination and then clicks the ``Request'' button
	\item  The system notifies the first taxi driver of the starting location's zone's queue. If there are no taxis in that zone, the system notifies the CAT\@. If a taxi driver rejects the request the system passes the request on to the next taxi driver in the queue and changes the first one's state to not available. If a taxi driver doesn't accept or reject the request within 1 minute, the system passes the request on to the next taxi driver in the queue and moves the first one to the end of the queue
	\item  The taxi driver accepts the request and goes to the starting location
	\item  The system changes the state of the taxi driver to not available and notifies the user of the code of the incoming taxi and the ETA 
	\item The taxi driver goes to the starting location to pick up the user
\end{itemize}
\textit{Postconditions:} The taxi driver is not available anymore and has been removed from the queue.
\\\textit{Exceptions:}
\begin{itemize}
	\item If a taxi driver doesn't get to the starting location of the request within the MAD, the system will pass on the request to the next taxi driver in the queue and will notify the user of the change with the new ETA and taxi code. If the queue is empty, the system will notify the CAT.\@
\end{itemize}

\subsubsection{Reserve a taxi}
\textit{Actors:} Registered user, taxi driver
\\\textit{Preconditions:} The user is on the homepage of the application %TODO va bene homepage anche per la mobile app?
\\\textit{Flow of Events:}
\begin{itemize}
	\item  The user clicks the ``Reserve a taxi'' button
	\item  The system shows the user a page where she can enter the starting location, destination and the meeting time of the reservation.
	\item  The user inserts the starting location, destination and meeting time and then clicks the ``Reserve'' button
	\item  Ten minutes before the meeting time, the system notifies the first taxi driver of the starting location's zone's queue. If there are no taxis in that zone, the system notifies the CAT\@. If a taxi driver rejects the reservation the system passes it on to the next taxi driver in the queue and changes the first one's state to not available. If a taxi driver doesn't accept or reject the reservation within 1 minute, the system passes it on to the next taxi driver in the queue and moves the first one to the end of the queue
	\item  The taxi driver accepts the reservation and goes to the starting location
	\item  The system changes the state of the taxi driver to not available and notifies the user of the code of the incoming taxi %TODO anche questo?-> and the ETA
	\item  The taxi driver goes to the starting location to pick up the user
\end{itemize}
\textit{Postconditions:} The taxi driver is not available anymore and has been removed from the queue. %TODO non sapevo che mettere %TODO vedi sopra
\\\textit{Exceptions:}
\begin{itemize} 
	\item If a taxi driver doesn't get to the starting location of the request within the MAD after the meeting time, the system will pass on the reservation to the next taxi driver in the queue and will notify the user of the change with the new%TODO--> ETA?<--- and taxi code.
	\item If the meeting time is not at least two hours after the reservation an error message will be displayed and the reservation won't be made.
\end{itemize}

\subsubsection{Cancel a reservation}
\textit{Actors:} Registered user
\\\textit{Preconditions:} The user is on the homepage of the application and has at least one reservation for which a taxi has not yet been allocated
\\\textit{Flow of Events:}
\begin{itemize}
	\item  The user clicks on the reservation she wants to cancel
	\item  The system shows the user a page with the details of the reservation
	\item  The user clicks the ``Cancel reservation'' button
	\item  The system shows a ``successful deletion'' message and brings the user back to her homepage
\end{itemize}
\textit{Postconditions:} The reservation has been cancelled.
\\\textit{Exceptions:}
\begin{itemize} 
	\item If a taxi is allocated while the user is cancelling the reservation the system will show an error message and won't cancel the reservation
\end{itemize}

\subsubsection{Cancel a request}
\textit{Actors:} Registered user
\\\textit{Preconditions:} The user is on the homepage of the application and has at least one request that hasn't been accepted by a taxi driver yet
\\\textit{Flow of Events:}
\begin{itemize}
	\item  The user clicks on the request she wants to cancel
	\item  The system shows the user a page with the details of the request
	\item  The user clicks the ``Cancel request'' button
	\item  The system shows a ``successful deletion'' message and brings the user back to her homepage
\end{itemize}
\textit{Postconditions:} The request has been cancelled.
\\\textit{Exceptions:}
\begin{itemize} 
	\item If a taxi driver accepts the request while the user is cancelling it the system will show an error message and won't cancel the request
\end{itemize}

\subsubsection{Registration}
\textit{Actors:} Guest
\\\textit{Preconditions:} The guest is on the homepage of the application
\\\textit{Flow of Events:}
\begin{itemize}
	\item  The guest clicks on the ``Register here'' button
	\item  The system shows a page where the guest can enter her username, password and email
	\item  The guest inserts the requested data and clicks the ``Register'' button
	\item  The system shows a ``successful registration'' message 
\end{itemize}
\textit{Postconditions:} The guest can now log in the system whenever she wants with the inserted credentials
\\\textit{Exceptions:}

\subsubsection{Login}
\textit{Actors:} Guest
\\\textit{Preconditions:} The guest is on the homepage of the application
\\\textit{Flow of Events:}
\begin{itemize}
	\item  The guest clicks on the ``Login'' button
	\item  The system shows a page where the guest can enter her username and password
	\item  The guest clicks the ``Login'' button
	\item  The system logs in the guest and shows her her page
\end{itemize}
\textit{Postconditions:} The guest is now logged in the system, thus becoming a taxi driver or a logged in user or an administrator depending on the inserted credentials.
\\\textit{Exceptions:} The inserted credentials are not valid so the guest is not logged in and an error message is shown

The following use case is described using logged in users as actors but it also valid for taxi drivers or administrators
\subsubsection{Logout}
\textit{Actors:} Logged in user
\\\textit{Preconditions:} The user is on the homepage of the application
\\\textit{Flow of Events:}
\begin{itemize}
	\item  The user clicks on the ``Logout'' button
	\item  The system logs the user out and brings her back to the homepage of the application for guests
\end{itemize}
\textit{Postconditions:} The user is now logged out of the system thus becoming a guest
\\\textit{Exceptions:}

The following use case is described using logged in users as actor but is also valid for guests
\subsubsection{Search for ETA}
\textit{Actors:} Logged in user
\\\textit{Preconditions:} The user is on the homepage of the application
\\\textit{Flow of Events:}
\begin{itemize}
	\item  The user clicks on the ``See estimated time of arrival'' button
	\item  The system shows a page where the user is prompted to insert the starting location from which the ETA must be calculated
	\item  The user inserts the starting location and clicks the ``See estimated time of arrival'' button
	\item  The system shows the ETA 
\end{itemize}
\textit{Postconditions:} The user has been shown the ETA
\\\textit{Exceptions:}

\subsubsection{Modify taxi driver}
\textit{Actors:} Administrator
\\\textit{Preconditions:} The administrator is on the homepage of the application
\\\textit{Flow of Events:}
\begin{itemize}
	\item  The administrator clicks on the ``Modify taxi driver'' button
	\item  The system shows the administrator the list of taxi drivers retrieved from the database
	\item  The administrator clicks on the taxi driver she wants to modify
	\item  The system shows the administrator a page where she can change the values of the fields of the taxi driver
	\item  The administrator changes the needed fields and then clicks the ``Save changes'' button
	\item  The system update the taxi driver's information with the new ones that have been inserted
\end{itemize}
\textit{Postconditions:} The taxi driver's information have been updated and she can no longer access the application with the old credentials but she must use the new ones
\\\textit{Exceptions:} %TODO possono essere non valide?

\subsubsection{Add taxi driver}
\textit{Actors:} Administrator
\\\textit{Preconditions:} The administrator is on the homepage of the application
\\\textit{Flow of Events:}
\begin{itemize}
	\item  The administrator clicks on the ``Add taxi driver'' button
	\item  The system shows the administrator a form where she can enter all the information of the taxi driver %TODO quali sono le info da inserire?
	\item  The administrator inserts the information then clicks on the ``Add taxi driver'' button
	\item  The system adds the taxi driver to the database
\end{itemize}
\textit{Postconditions:} The taxi driver's information have been added to the database and she can now log in the application as a taxi driver
\\\textit{Exceptions:}

\subsubsection{Delete taxi driver}
\textit{Actors:} Administrator
\\\textit{Preconditions:} The administrator is on the homepage of the application
\\\textit{Flow of Events:}
\begin{itemize}
	\item  The administrator clicks on the ``Delete taxi driver'' button
	\item  The system shows the administrator the list of taxi drivers retrieved from the database
	\item  The administrator clicks on the taxi driver she wants to delete
	\item  The system shows the administrator a page with the details of the taxi driver
	\item  The administrator clicks the ``Delete taxi driver'' button
	\item  The system deletes the taxi driver from the database
\end{itemize}
\textit{Postconditions:} The taxi driver's information have been deleted from the database and she can no longer access the application with those credentials
\\\textit{Exceptions:}
\clearpage
\section{Alloy}
\subsection{Model}
\texttt{\-     \ \ 1	\qquad {\color{blue}module} TAXInseconds\\
\-     \ \ 2	\qquad {\color{green}//TAXIES}\\
\-     \ \ 3	\qquad {\color{blue}abstract} {\color{blue}sig} Taxi\{\}\\
\-     \ \ 4	\qquad {\color{blue}sig} AvailableTaxi {\color{blue}extends} Taxi\{\\
\-     \ \ 5	\qquad \-\qquad {\color{green}//For the queues}\\
\-     \ \ 6	\qquad \-\qquad nextTaxi:lone AvailableTaxi, \\
\-     \ \ 7	\qquad \-\qquad {\color{green}//For the service}\\
\-     \ \ 8	\qquad \-\qquad serve:lone ActiveClient\\
\-     \ \ 9	\qquad \}\\
\-    \ 10	\qquad {\color{blue}sig} InactiveTaxi {\color{blue}extends} Taxi\{\\
\-    \ 11	\qquad \-\qquad refused:lone ActiveClient\\
\-    \ 12	\qquad \}\\
\-    \ 13	\qquad {\color{blue}sig} TaxiQueue\{root:AvailableTaxi\}\\
\-    \ 14	\qquad \\
\-    \ 15	\qquad {\color{green}//A Taxi can't be his next}\\
\-    \ 16	\qquad {\color{blue}fact} nextTaxiNotReflexive \{ \\
\-    \ 17	\qquad \-\qquad {\color{blue}no} t:AvailableTaxi| t = t.nextTaxi \\
\-    \ 18	\qquad \}\\
\-    \ 19	\qquad {\color{green}//A Taxi can't be one of his followers in the queue}\\
\-    \ 20	\qquad {\color{blue}fact} nextTaxiNotCyclic \{\\
\-    \ 21	\qquad \-\qquad {\color{blue}no} t:AvailableTaxi | t {\color{blue}in} t.\string^nextTaxi\\
\-    \ 22	\qquad \} \\
\-    \ 23	\qquad {\color{green}//If a taxi is active he must be in exactly one queue}\\
\-    \ 24	\qquad {\color{blue}fact} allAvailableTaxiesBelongToOneQueue \{\\
\-    \ 25	\qquad \-\qquad {\color{blue}all} t:AvailableTaxi | {\color{blue}one} q:TaxiQueue | t {\color{blue}in} q.root.*nextTaxi\\
\-    \ 26	\qquad \}\\
\-    \ 27	\qquad \\
\-    \ 28	\qquad {\color{green}//Domain Assumption}\\
\-    \ 29	\qquad {\color{blue}fact} thereIsAtLeastOneFreeTaxi\{\\
\-    \ 30	\qquad \-\qquad {\color{blue}some} t:AvailableTaxi | {\color{blue}no} c:ActiveClient |\\
\-    \ 31	\qquad \-\qquad t.serve = c\\
\-    \ 32	\qquad \}\\
\-    \ 33	\qquad \\
\-    \ 34	\qquad {\color{green}//CLIENTS}\\
\-    \ 35	\qquad {\color{blue}abstract} {\color{blue}sig} Client\{\}\\
\-    \ 36	\qquad {\color{blue}sig} ActiveClient {\color{blue}extends} Client\{\\
\-    \ 37	\qquad \-\qquad {\color{green}//For the queue}\\
\-    \ 38	\qquad \-\qquad nextClient:lone ActiveClient\\
\-    \ 39	\qquad \}\\
\-    \ 40	\qquad {\color{blue}sig} nonActiveClient {\color{blue}extends} Client\{\\
\-    \ 41	\qquad \-\qquad {\color{green}//For reservations}\\
\-    \ 42	\qquad \-\qquad reserved: {\color{blue}lone} Area\\
\-    \ 43	\qquad \}\\
\-    \ 44	\qquad {\color{blue}sig} ClientQueue\{root:ActiveClient\}\\
\-    \ 45	\qquad \\
\-    \ 46	\qquad {\color{green}//A Client can't be his next}\\
\-    \ 47	\qquad {\color{blue}fact} nextClientNotReflexive \{ \\
\-    \ 48	\qquad \-\qquad {\color{blue}no} c:ActiveClient| c= c.nextClient\\
\-    \ 49	\qquad \}\\
\-    \ 50	\qquad {\color{green}//A Client can't be one of his followers in the queue}\\
\-    \ 51	\qquad {\color{blue}fact} nextClientNotCyclic \{\\
\-    \ 52	\qquad \-\qquad {\color{blue}no} c:ActiveClient| c {\color{blue}in} c.\string^nextClient\\
\-    \ 53	\qquad \} \\
\-    \ 54	\qquad {\color{green}//If a client is Waiting for a Taxi}\\
\-    \ 55	\qquad {\color{green}//he must be in exactly one queue}\\
\-    \ 56	\qquad {\color{blue}fact} allActiveClientsBelongToOneQueue \{\\
\-    \ 57	\qquad \-\qquad {\color{blue}all} c:ActiveClient| {\color{blue}one} q:ClientQueue| c {\color{blue}in} q.root.*nextClient\\
\-    \ 58	\qquad \}\\
\-    \ 59	\qquad \\
\-    \ 60	\qquad {\color{green}//All clients must have a taxi serving them}\\
\-    \ 61	\qquad {\color{blue}fact} allClientsAreServed\{\\
\-    \ 62	\qquad \-\qquad {\color{blue}all} c:ActiveClient |\\
\-    \ 63	\qquad \-\qquad {\color{blue}some} t:AvailableTaxi |\\
\-    \ 64	\qquad \-\qquad c=t.serve\\
\-    \ 65	\qquad \}\\
\-    \ 66	\qquad \\
\-    \ 67	\qquad {\color{green}//AREAS}\\
\-    \ 68	\qquad {\color{blue}sig} Area\{\\
\-    \ 69	\qquad \-\qquad taxis: {\color{blue}one} TaxiQueue, \\
\-    \ 70	\qquad \-\qquad clients: {\color{blue}one} ClientQueue\\
\-    \ 71	\qquad \}\\
\-    \ 72	\qquad {\color{blue}fact} oneQueueoneArea\{\\
\-    \ 73	\qquad \-\qquad {\color{blue}no} c:ClientQueue | {\color{blue}some} disjoint a,a':Area | \\
\-    \ 74	\qquad \-\qquad \-\qquad c=a.clients  {\color{blue}and}  c=a'.clients\\
\-    \ 75	\qquad \-\qquad {\color{blue}no} t:TaxiQueue | {\color{blue}some} disjoint a,a':Area |\\
\-    \ 76	\qquad \-\qquad \-\qquad  t=a.taxis  {\color{blue}and}  t=a'.taxis\\
\-    \ 77	\qquad \}\\
\-    \ 78	\qquad \\
\-    \ 79	\qquad {\color{green}//All queues must be connected to an Area}\\
\-    \ 80	\qquad {\color{blue}fact} allQueuesInAreas\{\\
\-    \ 81	\qquad \-\qquad {\color{blue}all} c:ClientQueue | {\color{blue}some} a:Area | c=a.clients\\
\-    \ 82	\qquad \-\qquad {\color{blue}all} t:TaxiQueue | {\color{blue}some} a:Area | t=a.taxis\\
\-    \ 83	\qquad \}\\
\-    \ 84	\qquad \\
\-    \ 85	\qquad {\color{green}//INTERACTIONS}\\
\-    \ 86	\qquad {\color{green}//Clients are served only by one taxi}\\
\-    \ 87	\qquad {\color{blue}fact} noClientsObiquity\{\\
\-    \ 88	\qquad \-\qquad {\color{blue}no} disjoint t,t':AvailableTaxi | t'.serve=t.serve\\
\-    \ 89	\qquad \}\\
\-    \ 90	\qquad \\
\-    \ 91	\qquad {\color{green}//Clients are served in order}\\
\-    \ 92	\qquad {\color{blue}fact} ClientsRespectQueues\{\\
\-    \ 93	\qquad \-\qquad {\color{blue}no} c:ActiveClient | {\color{blue}some} t:AvailableTaxi | \\
\-    \ 94	\qquad \-\qquad c=t.serve {\color{blue}and} no t':AvailableTaxi | t'.serve=c.~nextClient\\
\-    \ 95	\qquad \}\\
\-    \ 96	\qquad \\
\-    \ 97	\qquad {\color{green}//Taxies are serving in order}\\
\-    \ 98	\qquad {\color{blue}fact} TaxisServeInOrder\{\\
\-    \ 99	\qquad \-\qquad {\color{blue}no} t,t':AvailableTaxi | {\color{blue}some} c:ActiveClient | \\
\-   100	\qquad \-\qquad t'=t.~nextTaxi {\color{blue}and} c=t.serve {\color{blue}and} no c':ActiveClient| c'=t'.serve\\
\-   101	\qquad \}\\
\-   102	\qquad \\
\-   103	\qquad {\color{green}//If an area has more clients than taxies }\\
\-   104	\qquad {\color{green}//all taxies must be serving}\\
\-   105	\qquad {\color{blue}fact} TaxiServeIfNeeded\{\\
\-   106	\qquad \-\qquad {\color{blue}no} t:AvailableTaxi | {\color{blue}some} a:Area | \\
\-   107	\qquad \-\qquad t {\color{blue}in} a.taxis.root.*nextTaxi and\\
\-   108	\qquad \-\qquad \#a.taxis.root.*nextTaxi <= \#a.clients.root.*nextClient \\
\-   109	\qquad \-\qquad and\-\qquad \#t.serve=0 \qquad \\
\-   110	\qquad \}\\
\-   111	\qquad \\
\-   112	\qquad {\color{green}//Serving local clients is preferrable}\\
\-   113	\qquad {\color{blue}fact} TaxiStayIfNeeded\{\\
\-   114	\qquad \-\qquad {\color{blue}no} t:AvailableTaxi | {\color{blue}some} a:Area | \\
\-   115	\qquad \-\qquad t {\color{blue}in} a.taxis.root.*nextTaxi and\\
\-   116	\qquad \-\qquad \#a.taxis.root.*nextTaxi <= \#a.clients.root.*nextClient \\
\-   117	\qquad \-\qquad and\-\qquad t.serve {\color{blue}not} in a.clients.root.*nextClient \\
\-   118	\qquad \}\\
\-   119	\qquad \\
\-   120	\qquad {\color{green}//FUNCTIONS}\\
\-   121	\qquad {\color{green}//Get who a taxi is serving}\\
\-   122	\qquad {\color{blue}fun} getTaxiClient[t:AvailableTaxi]: {\color{blue}lone} ActiveClient\{\\
\-   123	\qquad \-\qquad t.serve\\
\-   124	\qquad \}\\
\-   125	\qquad \\
\-   126	\qquad {\color{green}//Get who serves a client}\\
\-   127	\qquad {\color{blue}fun} getClientServer[c:ActiveClient]: {\color{blue}lone} AvailableTaxi\{\\
\-   128	\qquad \-\qquad c.~serve\\
\-   129	\qquad \}\\
\-   130	\qquad {\color{green}//Get Queues for an area}\\
\-   131	\qquad {\color{blue}fun} getTaxisInArea[a:Area]: {\color{blue}set} AvailableTaxi\{\\
\-   132	\qquad \-\qquad a.taxis.root.*nextTaxi\\
\-   133	\qquad \}\\
\-   134	\qquad {\color{blue}fun} getActiveClientsInArea[a:Area]: {\color{blue}set} ActiveClient\{\\
\-   135	\qquad \-\qquad a.clients.root.*nextClient\\
\-   136	\qquad \}\\
\-   137	\qquad \\
\-   138	\qquad {\color{green}//ASSERTIONS}\\
\-   139	\qquad {\color{green}//Taxies cross areas only if the client they are serving  }\\
\-   140	\qquad {\color{green}//is in an area without taxies}\\
\-   141	\qquad {\color{blue}assert} TaxisRespectAreas1 \qquad \{\\
\-   142	\qquad \-\qquad {\color{blue}all} t:AvailableTaxi | {\color{blue}some} ca,ta:Area | \\
\-   143	\qquad \-\qquad {\color{blue}some} c:ActiveClient | \\
\-   144	\qquad \-\qquad c {\color{blue}in} ca.clients.root.*nextClient and\\
\-   145	\qquad \-\qquad t {\color{blue}in} ta.taxis.root.*nextTaxi and\\
\-   146	\qquad \-\qquad c=t.serve {\color{blue}and} \\
\-   147	\qquad \-\qquad ca!=ta {\color{blue}implies} \\
\-   148	\qquad \-\qquad \#ca.taxis.root.*nextTaxi = 0\\
\-   149	\qquad \}\\
\-   150	\qquad {\color{blue}check} TaxisRespectAreas1 \qquad {\color{blue}for} 3\\
\-   151	\qquad \\
\-   152	\qquad {\color{green}//Taxies cross areas only if the client they are serving  }\\
\-   153	\qquad {\color{green}//is in an area with less taxies than clients}\\
\-   154	\qquad {\color{blue}assert} TaxisRespectAreas2 \qquad \{\\
\-   155	\qquad \-\qquad {\color{blue}all} t:AvailableTaxi | {\color{blue}some} ca,ta:Area | \\
\-   156	\qquad \-\qquad {\color{blue}some} c:ActiveClient | \\
\-   157	\qquad \-\qquad c {\color{blue}in} ca.clients.root.*nextClient and\\
\-   158	\qquad \-\qquad t {\color{blue}in} ta.taxis.root.*nextTaxi and\\
\-   159	\qquad \-\qquad c=t.serve {\color{blue}and} \\
\-   160	\qquad \-\qquad ca!=ta {\color{blue}implies} \\
\-   161	\qquad \-\qquad \#ca.taxis.root.*nextTaxi < \#ca.clients.root.*nextClient\\
\-   162	\qquad \}\\
\-   163	\qquad \\
\-   164	\qquad {\color{blue}check} TaxisRespectAreas2 \qquad {\color{blue}for} 3\\
\-   165	\qquad \\
\-   166	\qquad {\color{blue}assert} ActiveClientsMustBeInOneArea \{\\
\-   167	\qquad \-\qquad {\color{blue}all} c:ActiveClient | {\color{blue}some} t:AvailableTaxi | \\
\-   168	\qquad \-\qquad {\color{blue}some} a:Area | \\
\-   169	\qquad \-\qquad c=t.serve {\color{blue}implies} c {\color{blue}in} a.clients.root.*nextClient\\
\-   170	\qquad \}\\
\-   171	\qquad {\color{blue}check} ActiveClientsMustBeInOneArea {\color{blue}for} 3\\
\-   172	\qquad \\
\-   173	\qquad {\color{blue}assert} TaxiQueuesAreRespected\{\\
\-   174	\qquad \-\qquad {\color{blue}no} t:AvailableTaxi | {\color{blue}some} t':AvailableTaxi | \\
\-   175	\qquad \-\qquad t' {\color{blue}in} t.*nextTaxi and\\
\-   176	\qquad \-\qquad \#t'.serve = 1 \qquad and\\
\-   177	\qquad \-\qquad \#t.serve = 0 \qquad \\
\-   178	\qquad \}\\
\-   179	\qquad \\
\-   180	\qquad {\color{blue}check} TaxiQueuesAreRespected {\color{blue}for} 3\\
\-   181	\qquad \\
\-   182	\qquad {\color{blue}assert} ClientQueuesAreRespected\{\\
\-   183	\qquad \-\qquad {\color{blue}all} c,c':ActiveClient | {\color{blue}some} t,t':AvailableTaxi | \\
\-   184	\qquad \-\qquad c' {\color{blue}in} c.*nextClient and\\
\-   185	\qquad \-\qquad c' {\color{blue}in} t.serve implies\\
\-   186	\qquad \-\qquad c {\color{blue}in} t'.serve\\
\-   187	\qquad \}\\
\-   188	\qquad \\
\-   189	\qquad {\color{blue}check} ClientQueuesAreRespected {\color{blue}for} 3\\
\-   190	\qquad \\
\-   191	\qquad {\color{green}//PREDICATES}\\
\-   192	\qquad \\
\-   193	\qquad {\color{blue}pred} OneAreaFewAgents\{\\
\-   194	\qquad \-\qquad \#AvailableTaxi = 2\\
\-   195	\qquad \-\qquad \#Area = 1\\
\-   196	\qquad \-\qquad \#ActiveClient = 1\\
\-   197	\qquad \}\\
\-   198	\qquad {\color{blue}run} OneAreaFewAgents\{\} {\color{blue}for} 2 \qquad \\
\-   199	\qquad \\
\-   200	\qquad \\
\-   201	\qquad {\color{blue}pred} ALotOfTaxis\{\\
\-   202	\qquad \-\qquad \#AvailableTaxi = 5\\
\-   203	\qquad \-\qquad \#Area = 1\\
\-   204	\qquad \-\qquad \#ActiveClient=1\\
\-   205	\qquad \}\\
\-   206	\qquad \\
\-   207	\qquad {\color{blue}run} ALotOfTaxis\{\} {\color{blue}for} 2 \qquad \\
\-   208	\qquad \\
\-   209	\qquad {\color{blue}pred} TwoAreas\{\\
\-   210	\qquad \-\qquad \#Area=2\\
\-   211	\qquad \}\\
\-   212	\qquad \\
\-   213	\qquad {\color{blue}run} TwoAreas\{\} {\color{blue}for} 2\\
\-   214	\qquad \\
\-   215	\qquad {\color{blue}pred} show\{\}\\
\-   216	\qquad {\color{blue}run} show\{\} {\color{blue}for} 3\\
}	
\subsection{Result}
\begin{center}
	\includegraphics[width=.9\textwidth,height=.9\textheight,keepaspectratio]{AlloyChecks}
\end{center}
\subsection{Worlds Generated}
The following worlds were generated changing parameters about arity of present signatures.\\
\begin{center}
	This is a clean example of two areas, each one with a single client, no unusual interaction occour.\\
	\includegraphics[width=.9\textwidth,height=.9\textheight,keepaspectratio]{2Areas}
\end{center}
\begin{center}
	This is a more complex example, here we can see how the system behaves and the model adjusts if there aren't enough taxies in an area in order to serve all the clients.
	\includegraphics[width=.9\textwidth,height=.9\textheight,keepaspectratio]{CrossAreaService}
\end{center}
\begin{center}
	In the following scenario we can see both a reservation of a client for a determined area and a refusal of a taxi for a client (now being served by an other taxi)
	\includegraphics[width=.9\textwidth,height=.9\textheight,keepaspectratio]{reserved-and-refused}
\end{center}
\end{document}
>>>>>>> origin/master


%	\begin{itemize}
%		\item  
%	\end{itemize}
