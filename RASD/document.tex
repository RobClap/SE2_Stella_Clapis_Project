\documentclass{article}
\usepackage{graphicx}
\graphicspath{{images/}}
\usepackage{booktabs}
\usepackage{verbatim}
\usepackage{listings}
\usepackage{underscore}
\setcounter{secnumdepth}{5}
\usepackage[bookmarks=true]{hyperref}
\author{Roberto Clapis (841859), Erica Stella (854443)} 
\date{\today}
\title{Politecnico di Milano
	\\A.A. 2015\@-\@2016
	\\Software Engineering 2: ``TAXInseconds''
	\\\textbf{R}equirements \textbf{A}nalysis \\and\\ \textbf{S}pecifications \textbf{D}ocument}
\hypersetup{pdftitle={Software Requirement Specification},    % title
	pdfauthor={Roberto Clapis, Erica Stella},                     % author
	pdfsubject={TeX and LaTeX},                        % subject of the document
	pdfkeywords={TeX, LaTeX, taxi, RASD, SoftwareEngineering2}, % list of keywords
	colorlinks=true,       % false: boxed links; true: colored links
	linkcolor=black,       % color of internal links
	citecolor=blue,       % color of links to bibliography
	filecolor=black,        % color of file links
	urlcolor=purple,        % color of external links
}
\begin{document}
\maketitle
\includegraphics{polimi-logo}
\clearpage
\tableofcontents
\clearpage

\section{Introduction}

\subsection{Purpose}
This document represent the Requirement Analysis and Specification Document (RASD). It aims at explaining the domain of the system to be developed and the system itself in terms of functional requirements, nonfunctional requirements and constraints. It also provides several models of the system and typical use cases. It is intended for all the developers who will have to implement the system, the testers who will have to determine if the requirements have been met and the system analysts who will have to write specifications for other system that will relate to this one. It is also intended as a contractual basis thus being legally binding.

\subsection{Actual System}
The government of the city wants to optimize its taxi service with a completely new application. Therefore, we assume there are no previous systems to take into account.

\subsection{Scope}
The aim of the project TAXINSECONDS is to provide a new application to optimize the taxi service of the city that will be accessible via browser, mobile or public APIs.
\\The application will be available to the users in web and mobile forms and will have public APIs in order to expand and improve the service with additional features. %TODO ripetuto due volte? %TODO togli pure
\\The city managed by TAXINSECONDS is divided in zones of 2 km\textsuperscript{2} each and every zone has its own queue of taxis. The queues are automatically computed by the system with the information it receives from the GPS of the taxis. %TODO non quello dei telefoni? come ti aspetti che comunichino i telefoni con le automobili?
\\Taxi drivers can be available or not. Only available taxi drivers can be in a queue. When a taxi driver changes her state from not available to available the system automatically adds her to the queue of the zone she is currently in, based on the information of the GPS of her taxi. 
\\Users that are not registered can only see the estimated time of arrival of the nearest taxi with TAXINSECONDS.\@
\\Registered users can also request a taxi or make a reservation for a taxi. Reservations can only be made at least two hours before the ride and must be done specifying the starting location, the destination and the meeting time. Requests, instead, only need the starting location and the destination.
\\When a request is made, the first taxi driver of the queue of the starting location's zone is notified for accepting or rejecting it. If the taxi driver rejects it her state is automatically put on unavailable by the system. If a taxi driver doesn't accept or reject the request within 1 minute, it will be passed on to the next taxi driver in the queue and the first one will be moved to the end of the queue. If there are no available taxis in the zone of the request the system will propagate the request to the closest available taxi.
\\When a request is accepted, the user that has made the request receives a notification from the system informing her of the code of the incoming taxi and the estimated time of arrival.
\\When a reservation is made, the system confirms it to the user and allocates a taxi 10 minutes before the meeting time. If a taxi for that zone is not available the closest available taxi will be notified. When a taxi driver accepts the reservation, the user receives from the system the code of the incoming taxi. If a taxi driver doesn't accept or reject the reservation within 1 minute, it will be passed on to the next taxi driver in the queue and the first one will be moved to the end of the queue.
\\Requests can be cancelled before they have been accepted by a taxi driver while reservation can be cancelled until 10 minutes before the meeting time.

\subsection{Goals}
\begin{itemize}
	\item Provide an easy way to request a taxi.
	\item Provide an easy way to reserve a taxi.
	\item Guarantee a fair management of the taxi queues.
	\item Create an extensible system that allows expansion and interactions with other services.
\end{itemize}

\subsection{Definition and Acronyms}

\subsubsection{Definitions}
\begin{itemize}
	\item \textit{Guest:} a person that has to sign up or log in the system.
	\item \textit{Secure Channel:} a communication channel to ensure privacy and autenticity for both the server and the clients 
	\item \textit{Logged in user:} a person that has already signed up and logged in the system.
	\item \textit{Administrator:} a person authorized to modify the list of taxi drivers stored by the system.
	\item \textit{Request:} a call from a registered user who needs a taxi immediately.
	\item \textit{Meeting time:} the date and time in which the registered user needs the taxi in case of reservation. %TODO ricordare che meeting time comprende anche la data -> erica polla
	\item \textit{Reservation:} a booking of a taxi at a certain meeting time.
	\item \textit{State of a taxi driver:} the state the taxi driver is currently in. It can be available or not available. Taxi drivers can be in a queue if and only if they're available.
	\item \textit{Closest available taxi:} if there are no taxis in the zone of the request or of the reservation, the system automatically finds the closest available taxi choosing the one with the smallest estimated time of arrival from the taxi queues of the other zones.
\end{itemize}

\subsubsection{Acronyms}
\begin{itemize}
	\item ETA:\@ estimated time of arrival: the time, estimated by the system, that the closest available taxi will take to get to the starting location of the ride.
	\item CAT:\@ closest available taxi (see definition in the previous paragraph).
	\item API:\@ application programming interface is a set of routines, protocols, and tools for building software applications on top of this one.
	\item MAD:\@ maximum allowed delay; the maximum time, calculated by the system according to its information about distance and traffic, that a taxi driver has to get to the starting location of a request. 
\end{itemize}
\subsection{Actors}
\begin{itemize}
	\item \textit{Guest}: guests are able to sign up, login or ask the system for an ETA.\@
	\item \textit{Registered users}: after successfully logging in, registered users can request or reserve taxis or ask the system for an ETA.\@
	\item \textit{Taxi drivers}: after successfully logging in, taxi drivers are able to set their current state as available or not and to accept or refuse requests.
	\item \textit{Administrator}: after successfully logging in, the administrator will be the only user allowed to edit the taxi drivers list stored by the system.
\end{itemize}

\subsection{References}
\begin{itemize}
	\item The \href{run:./external_references/assignments.pdf}{document} with the assignment for the project
	\item The \href{run:./external_references/assignments.pdf}{IEEE Standard for SRS } 
\end{itemize}
\subsection{Overview}
This document is structured in three parts:
\begin{itemize}
	\item Introduction: gives an high-level description of the software purposes and context.
	\item Overall Description: gives a general description of the application, focusing on the context of the system, going in details about domain assumptions and constraintS. The aim of this section is to provide a context to the whole project and show its integration with the real world.
	\item Specific Requirements: this section contains all of the software requirements to a level of detail aimed to be enough to design a system to satisfy said requirements, and testers to test that the system actually satisfies them. It also contains the detailed description of the possible interactions between the system and the world with a simulation and preview of the expected response of the system with given stimulation. (Details are given with Alloy specifications and UML diagrams)
\end{itemize}


\section{Overall description}
\subsection{Product perspective}
The TAXINSECONDS application will be released as a web application and as a mobile application. 
There are no existing systems to integrate it with. 
\\It will provide a total of 4 main interfaces:
\begin{itemize}
	\item For both type of users
		\begin{itemize}
			\item Registered users
			\item Guests
		\end{itemize}
	\item For taxi drivers 
	\item For administrators
	\item A non graphical interface for APIs 
\end{itemize}

\subsubsection{User Interfaces}
%TODO Erica
\subsubsection{System interfaces}
\begin{comment}
This should list each system interface and identify the functionality of the software to accomplish the system requirement and the interface description to match the system.
\end{comment} %TODO non ho capito davvero cosa sia questa %non dovrebbe essere semplicemente il titolo che raccoglie el successive? in teoria no, è una subsection come le altre %TODO ottimo, e che ci facciamo?
\paragraph{Hardware Interfaces}
\begin{itemize}
\item Owned by the users and taxi drivers:
	\begin{itemize}
		\item Any device running Android 4.0+ or iOS 6+ (GPS capability will make more functionalities available)
		\item Any computer able to run an HTML5-compatible browser 
	\end{itemize}
\item Owned by the company:
	\begin{itemize}
		\item The server on which the core of the application will run, and to which the applications, 
			the web UI and the API-related clients will connect to.
		\item Two machines optimized for a stateful firewall use
		\item A machine with the DBMS 
	\end{itemize}
\end{itemize}
\paragraph{Software Interfaces}
\begin{itemize}
	\item Database Management System (DBMS):
		\begin{itemize}
			\item Name: MySQL.\@
			\item Version: 5.1.73 
			\item Source: https://www.mysql.com/
		\end{itemize}
	\item HTTP/HTTPS server:
		\begin{itemize}
			\item Name: Nginx
			\item Version: 1.8.0 
			\item Source: http://nginx.org/
		\end{itemize}
	\item PHP interpreter:
		\begin{itemize}
			\item Name: PHP
			\item Version: 5.3.3 
			\item Source: https://secure.php.net/
		\end{itemize}
	\item Operating System: 
		\begin{itemize}
			\item Name: CentOS
			\item Version: 7.1--1503 
			\item Source: https://www.centos.org/ 
		\end{itemize}
\end{itemize}
\paragraph{Communication Interfaces}
\begin{center}
	\begin{tabular}{*{4}{c}}
	\toprule
    Protocol & Application layer Protocol & Port & Scope \\
	\midrule
	TCP & HTTP & 80 & Upgrade to a secure connection over HTTPS \\ 
    TCP & HTTPS & 443 & The web interface or the mobile apps \\ 
    TCP & HTTPS/JSON & 443 & The APIs \\ 
    TCP & HTTPS & 443 & The web interface or the mobile apps \\ 
    TCP & DBMS over SSL & 3306 & Communication between the webserver and the DBMS \\ 
    \bottomrule
    \end{tabular}
\end{center}
\paragraph{Memory constraints} 
\begin{itemize}
	\item Primary memory:
		\begin{itemize}
			\item for both taxi drivers' and clients's mobile devices at least 500MB 
			\item for the web application 1GB or more is suggested
			\item for the server it is suggested to use a cloud service in order to resize memory according to traffic
		\end{itemize}
	\item Secondary memory:
		\begin{itemize}
			\item mobile devices will need to have 50MB of free space on the device
			\item the web application requires no secondary memory 
			\item for the server it is suggested to use a cloud service in order to resize memory according to traffic
		\end{itemize}
\end{itemize}
\paragraph{Operations} 
%TODO
\begin{comment}
This should specify the normal and special operations required by the user such as
The various modes of operations in the user organization (e.g., user-initiated operations);
Periods of interactive operations and periods of unattended operations;
Data processing support functions;
Backup and recovery operations.
\end{comment}
\subsection{Product functions}
\subsection{User characteristics}
The TAXINSECONDS application is intended for all users who are at least 18 years old.

%\subsection{Constraints}
%TODO non credo che il nostro progetto abbia constraints tipo quelli elencati nello standard anche perchè credo che i requisiti di reliability tipo del dbms siano nei requisiti non funzionali, no?  %Mi scazza che qua chieda cose tecnicissime come aCK NACK e il resto sia vaghissimo cmq no, direi che la balziamo %approvato
\subsection{Assumptions and dependencies}
\subsubsection{Domain Assumptions}
\begin{itemize}
	\item All taxi drivers who intend to use the service will have a mobile phone with one of the supported mobile OSs
	\item All taxi drivers will have a phone with active GPS functionality %TODO mettiamo i gps delle macchine, no?	%NO, come li fai comunicare con la APP?
	\item The taxi drivers will grant the system the rights to handle their taxi codes
	\item Taxi drivers' phones will always have an internet connection while TAXINSECONDS is running
	\item Users will have access to the internet %TODO dici che serve? %Si, come fanno a rpenotare se no? piccioni?
	\item In order to help the client to select the current location a phone with GPS capability will be required %TODO non credo sia una domain assumption %cos'è allora?
	\item Users will enter a valid email address during registration 
	\item Users will enter valid credit card data during registration
	\item Users allow the app to access their credit in order to pay for the service
	\item The Owner of the app will have to build or rent an always-on DBMS and host for the Server-Side part of the app %per ora lascialo 
	\item Users who have requested or reserved a taxi will always be present when the taxi arrives.
	\item There is always at least an available taxi to fulfill a request or a reservation in the whole city.
\end{itemize}
\clearpage
\section{Specific requirements}
\subsection{Functional Requirements}
On the user side:\@
\begin{itemize}
	\item For non logged in users (on both the WEB and Mobile interface):
		\begin{itemize}
			\item The system will be able to calculate and show the ETA.\@ 
			\item The system will provide a registration functionality 
			\item The system will provide a login functionality
		\end{itemize}
	\item For logged in users (on both the WEB and Mobile interface):
		\begin{itemize}
			\item The system will be able to calculate and show the ETA.\@ 
%TODO questa è una nonfunctional requirement			\item The system will store the username, password, email and credit card data of every user %TODO non trovo più dove c'erano scritte le altre cose che devi scrivere nel login e che deve salvare-->sotto ho scritto magari di metterle in un posto comune con altre funct
			\item The system will provide a functionality to request a taxi at the given starting location 
			\item The system will provide a functionality to reserve a taxi at the given starting location and meeting time
			\item The system will provide a functionality to modify the personal data of the user
			\item The system will provide a logout functionality
			\item The system will provide a functionality to notify users of the code of the incoming taxi
			\item The system will provide a functionality to notify users of the ETA of the incoming taxi
			\item The system will provide a logout functionality
		\end{itemize}
	\item Through the API:\@ %TODO c'è qualquadra che non cosa, ti va se ne parliamo quando hai tempo?
		\begin{itemize}
			\item The system will be able to calculate and show the ETA.\@ 
			\item Establish a Secure Channel 
			\item Using a Secure Channel and valid credentials:\@
				\begin{itemize}
					\item Place a request for a given starting location
					\item Place a reservation for given starting location and meeting time
					\item Require the system to send a push message for updates about the status of a previous request
				\end{itemize}
		\end{itemize}
\end{itemize}
On the taxi driver side:\@ %TODO (there will be no Web interface in this case) <- non è meglio specificarlo dove abbiamo detto che interfacce abbiamo? %TODO io non ho fatto la parte di interfacce, se lo vuoi specificare specificalo
\begin{itemize}
	\item If credentials have been invalidated a ``reset password'' procedure will be mandatory %TODO ?.? ti spiego luned\label{}<++>
	\item The system will provide a functionality to switch the state of the taxi driver from available to not available or vice versa
	\item The system will notify the taxi driver of an incoming request showing the starting location
	\item The system will notify the taxi driver of a reservation showing the starting location and the meeting time
	\item The system will provide a functionality to allow taxi drivers to accept or reject reservations and requests
	\item Handle timeout (if reaching the client takes too much time) %TODO questa funzione e le altre per gestire le eccezioni secondo me sono da spostare in una sottosezione utilities insieme ai pagamenti, al fatto che controlla la posizione dei taxi driver, al computare le queue e metterei cosa salva di ogni utente %TODO non ho capito un cazzo
	\item The system will provide a logout functionality
\end{itemize}
	\item Through the API:\@ %TODO le api mi sono oscure T.T %TODO te le spiego lunedi
		\begin{itemize}
			\item Establish a Secure Channel
			\item Through a Secure Channel and with valid credentials:\@
				\begin{itemize}
					\item Toggle the driver state
					\item Send a notification when a call is made for the driver
					\item Accept the answer to the call, whether the answer is Acceptance or Refusal
					\item Handle disconnection
					\item Handle timeout (if no taxi reach the client in time)
				\end{itemize}
			\item 
		\end{itemize}
On the admin side:\@
		\begin{itemize}
			\item Add a new taxi driver
			\item Remove a taxi driver
			\item Invalidate a taxi driver's credentials %TODO se rimuove il taxista dal db non può più accedere per cui non deve invalidare le sue credenziali-->dobbiamo scrivere da qualche parte che l'accesso viene fatto su credenziali? %TODO invalidarle, non eliminarlo, significa che sospettano che qualcuno le abbia rubate, quindi gliele annullano, e lui deve selezionare una nuova pw
			\item Set the area of the map handled by the system %TODO davvero? tra l'altro, dove mettiamo le cose che gli input sono suggeriti? nelle funct degli utenti? %TODO vedi ``USABILITY'' in nonfunct req. (non sono cosi' ovvie a quanto pare)
			\item Modify registration data
			\item Log out
		\end{itemize}
\subsection{Non functional requirements}
\begin{itemize}
	\item \textit{Cross platform} %TODO sono un po in dubbio su questo req %TODO ovvero?
		\begin{itemize}
			\item There will be a UI for mobile and one for the Web platform 
			\item At least Android 4.0+ and iOS 6+ will be supported for the mobile version 
			\item At least Edge, Chrome and Firefox will have to be supported
			\item The web application will have to use HTML5 %TODO questo non credo sia un nonfunctional req %TODO allora cosa è?
		\end{itemize}
	\item \textit{Availability}
		\begin{itemize}
			\item The application must always be available
			\item If a query is taking some time a spinner will be shown, the app must never freeze %TODO questo non è un req di availability %TODO quindi cosa è?
			\item The system must store all of its data in an always-reachable database
			\item Regular backups will be made in order to reduce or prevent data loss
		\end{itemize}
	\item \textit{Usability} %TODO credo che usability non sia un req, nel senso che da noi han detto che è una di quelle cose che dici "ma va?" %TODO ma se sopra mi hai detto di mettere che l'utente è aiutato con posizioni suggerite..
		\begin{itemize}
			\item The UIs have to be user friendly and with a few, clear functionalities
			\item The app will have few activities, mainly the login, the register and the call form.
			\item When inserting a location for a reservation the user will be helped with autocompletion for locations.
		\end{itemize}
	\item \textit{Security}
		\begin{itemize}
			\item In no situation sensible data will pass through an insecure channel
			\item No one should be able to impersonate the taxi drivers, the clients or the admin without proper authentication %TODO mmmhhh questa non mi convince %TODO preferisci che sia vero il contrario?
		\end{itemize}
%	\item 
\end{itemize}
\subsection{Scenarios}
\subsubsection{Scenario 1}
Blair The Witch had to take her magical broom to the mechanic for the annual revision but she needs to go shopping to refill her stockpile of frog's tails. Her friend Mizune has told her about TAXInseconds, so she decides to give it a try. After downloading it on her smartphone, she signs up compiling the registration form with her username, password, email and credit card data. Now she can complain about how slow car-based transports are!
\subsubsection{Scenario 2}
Suzuka is having a date tonight but, unfortunately, her car doesn't want to start. After several failures, she decides to use TAXINSECONDS\@. After logging in, she requests a taxi specifying her home as the starting location and the restaurant's address as the destination. The system notifies Takeshi, the first taxi driver of the queue in Suzuka's zone, of the request. Takeshi accepts the request and the system sends Suzuka the code of Takeshi's taxi and the ETA so she can finally get to her date.
\subsubsection{Scenario 3}
Ash Ketchum has to start his new adventure in Hoenn tomorrow but his mother is busy cleaning the house with Mr.\ Mime, so she can't take him to the airport to meet Prof.\ Oak. Ash decides to use once again TAXINSECONDS to reserve a taxi. After logging in, he makes a reservation for a taxi specifying the starting location as Pallet town, the destination and the meeting time. The morning after, 10 minutes before the meeting time, the system allocates Brock's taxi for the reservation and sends his code to Ash so that he knows who he'll meet to start his new adventure. 
\subsubsection{Scenario 4}
Donald Duck is ready to start his first day as a taxi driver in Paperopoli. He jumps on his car, logs in TAXINSECONDS and changes his state to availables. Unfortunately he doesn't know that, in his zone, there are 15 taxis in the queue before him, so he'll have to be patient for some time.
\subsubsection{Scenario 5}
After quite some time, Daisy requests a taxi and it's finally Donald's turn! The system notifies him but\ldots it's breakfast time and Donald's having a cappuccino in a very crowded bar. He doesn't hear the notification popping on his phone so, after 1 minute, the system forwards Daisy's request to the next taxi driver in the queue and changes Donald's state to not available.
\subsubsection{Scenario 6}
Mickey Mouse is 

\subsection{Use cases}
\subsubsection{Request a taxi}
\textit{Actors:} Registered user, taxi driver
\textit{Preconditions:} The user is on the homepage of the application
\textit{Flow of Events:}
	\begin{itemize}
		\item  The user clicks the ``Request a taxi'' button
		\item  The system shows the user a page where she can enter the starting location and destination of the request
		\item  The user inserts the starting location and the destination and then clicks the ``Request'' button
		\item  The system notifies the first taxi driver of the starting location's zone's queue. If there are no taxis in that zone, the system notifies the CAT\@. If a taxi driver rejects the request the system passes the request on to the next taxi driver in the queue and changes the first one's state to not available. If a taxi driver doesn't accept or reject the request within 1 minute, the system passes the request on to the next taxi driver in the queue and moves the first one to the end of the queue
		\item  The taxi driver accepts the request and goes to the starting location
		\item  The system changes the state of the taxi driver to not available and notifies the user of the code of the incoming taxi and the ETA %TODO perchè lo mette come non disponibile? semplicemente lo toglie dalla coda, no?
		\item  When the taxi driver arrives at the starting location, according to her GPS information, the system automatically makes the user pay the minimum fare 
	\end{itemize}
\textit{Postconditions:} The taxi driver is not available anymore and has been removed from the queue. The user has been charged with the minimum fare. %perchè il taxi non è disponibile? più disponibile di cosi'! sta servendo!
\textit{Exceptions:}
	\begin{itemize}
		\item If there are no available taxis in the entire city an error message is shown and the request is not made
		\item If a taxi driver doesn't get to the starting location of the request within the MAD the system will pass on the request to the next taxi driver in the queue and will notify the user of the change with the new ETA and taxi code.
	\end{itemize}
	
	\subsubsection{Reserve a taxi}
	\textit{Actors:} Registered user, taxi driver
	\textit{Preconditions:} The user is on the homepage of the application %TODO va bene homepage anche per la mobile app?
	\textit{Flow of Events:}
	\begin{itemize}
		\item  The user clicks the ``Reserve a taxi'' button
		\item  The system shows the user a page where she can enter the starting location, destination and the meeting time of the reservation.
		\item  The user inserts the starting location, destination and meeting time and then clicks the ``Reserve'' button
		\item  Ten minutes before the meeting time, the system notifies the first taxi driver of the starting location's zone's queue. If there are no taxis in that zone, the system notifies the CAT\@. If a taxi driver rejects the reservation the system passes it on to the next taxi driver in the queue and changes the first one's state to not available. If a taxi driver doesn't accept or reject the reservation within 1 minute, the system passes it on to the next taxi driver in the queue and moves the first one to the end of the queue
		\item  The taxi driver accepts the reservation and goes to the starting location
		\item  The system changes the state of the taxi driver to not available and notifies the user of the code of the incoming taxi and the ETA
		\item  When the taxi driver arrives at the starting location, according to her GPS information, the system automatically makes the user pay the minimum fare
	\end{itemize}
	\textit{Postconditions:} The taxi driver is not available anymore and has been removed from the queue. The user has been charged with the minimum fare. %TODO non sapevo che mettere %TODO vedi sopra
	\textit{Exceptions:}
	\begin{itemize}
		\item If there are no available taxis in the entire city an error message is shown and\ldots %TODO dobbiamo pensare a come fare per garantire comunque un taxi per le reservations altrimenti è inutile fare la reservation
		\item If a taxi driver doesn't get to the starting location of the request within the MAD after the meeting time, the system will pass on the reservation to the next taxi driver in the queue and will notify the user of the change with the new ETA and taxi code.
		\item If the meeting time is not at least two hours after the reservation an error message will be displayed and the reservation won't be made.
	\end{itemize}
\end{document}


%
%	\begin{itemize}
%		\item  
%	\end{itemize}
