\documentclass{article}
\usepackage{adjustbox}
\usepackage{float}
\usepackage{textcomp}
\usepackage{graphicx}
\graphicspath{{images/}}
\usepackage{booktabs}
\usepackage{color}
\usepackage{verbatim}
\usepackage{listings}

\definecolor{dkgreen}{rgb}{0,0.6,0}
\definecolor{gray}{rgb}{0.5,0.5,0.5}
\definecolor{mauve}{rgb}{0.58,0,0.82}

\lstset{frame=tb,
	language=Java,
	aboveskip=3mm,
	belowskip=3mm,
	showstringspaces=false,
	columns=flexible,
	basicstyle={\small\ttfamily},
	numbers=none,
	numberstyle=\tiny\color{gray},
	keywordstyle=\color{blue},
	commentstyle=\color{dkgreen},
	stringstyle=\color{mauve},
	breaklines=true,
	breakatwhitespace=true,
	tabsize=3
}

\usepackage{underscore}
\setcounter{secnumdepth}{5}
\usepackage[bookmarks=true]{hyperref}
\author{Roberto Clapis (841859), Erica Stella (854443)} 
\date{\today}
\title{Politecnico di Milano
	\\A.A. 2015\@-\@2016
	\\Software Engineering 2: ``myTaxiService''
	\\\textbf{C}ode \textbf{I}nspection}
\hypersetup{pdftitle={Code Inspection},    % title
	pdfauthor={Roberto Clapis, Erica Stella},                     % author
	pdfsubject={Code Inspection},                        % subject of the document
	pdfkeywords={TeX, LaTeX, taxi, Inspection, SoftwareEngineering2}, % list of keywords
	colorlinks=true,       % false: boxed links; true: colored links
	linkcolor=black,       % color of internal links
	citecolor=blue,       % color of links to bibliography
	filecolor=black,        % color of file links
	urlcolor=purple,        % color of external links
}
\begin{document}
	\maketitle
	\begin{center}
		\includegraphics{polimi-logo}
	\end{center}
	\clearpage
	\tableofcontents
	\clearpage
	
\section{Assigned Class}
%state the namespace pattern and name of the classes	
%that were assigned	to you	
 �	
\section{Functional Role of Class ConnectorDeployer}
%elaborate on the functional role you have
%identified for the class cluster that was
%assigned to you, also, elaborate on how
%you managed to understand this role and
%provide necessary evidence, e.g., 
%javadoc, diagrams, etc.

\section{Found Issues}
%report the classes/code fragments that 
%do not fulfill some points in the checklist.
%Explain which point is not fulfilled
%and why
%BEGINROBSECTION
\subsection{Naming Conventions}
\subsubsection{1}
Class name is meaningful;\\
No interfaces are in the file;\\
Method names are meaningful, but 2 problems emerged:
\begin{itemize}
\item it is suggested to change "deleteRAConfig" in "deleteResourceAdapterConfig";
\item it is suggested to change the ``event'' method name in something more clear, or at least document it.
\end{itemize}
Class variables are meaningful.\\
Most of Method variables are meaningful but 2 characters variables are confusing, rename them.\\ 
Constants names are meaningful but the "EAR" constant should be renamed in "ENTERPRISE_ARCHIVE" to improve readability;\\ 

\subsubsection{2}
Some one-character variables were found, but they were all "e" for exceptions.
We considered acceptable to have exceptions in catch blocks named "e" since they are throwaway variables and they have a very limited scope length.
\subsubsection{3}
The file only contains one class and it respects the naming convention.
\subsubsection{4}
No interface is declared in the assigned file.
\subsubsection{5}
All the methods respect the naming convention.
\subsubsection{6}
The convection is respected, but the variable "clh" has a meaningless name, because an acronym is used, but as a 3 letter lowercase word, which can be confusing. It is suggested to rename the variable clHierarchy or classLoaderHierarchy to improve readability. %TODO ma allora questo non � da mettere prima nel punto 1? 
%Se non ti piace qua spostalo pure
\subsubsection{7}
The constants respect the naming convention.
\subsection{Indention}
\subsubsection{8}
Indention is coherent, 4 or multiples of 4 spaces are used consistently.
\subsubsection{9}
line 518 uses tabs to indent.\\ %TODO non c'� lo stesso ignore sia in validationmapping che in registerbean? %ma � stato indentato con gli spazi
\begin{lstlisting}
513	                        try {
514	                            if (inputStream != null) {
515	                                inputStream.close();
516	                            }
517	                        } catch (Exception e) {
518	                            // ignore ?
519	                        }
\end{lstlisting}
\subsection{Braces}
\subsubsection{10}
Consistent bracing style is used.
\subsubsection{11}
Curly braces are always used for conditional statements.
\subsection{File Organization}
\subsubsection{12}
A blank line is missing before line 55 to divide imports of different domains.
There are 2 meaingless blank lines (216 217) that break coherence in spacing.
Blank lines 299, 342, 373 are meaningless.
\subsubsection{13}
%TODO e non ce n'� nessuna da segnalare che avrebbe senso splittata? Onestamente non ne ho trovate, secondo me l'hanno fatto fare a del codice
%Se vuoi guardare sono in sh/longerthan80 per� ci ho gi� dato un'occhiata io e non mi pare.
There are 74 lines longer than 80 chars. Most of them make sense as they are since splitting the line would just reduce readability. Some lines in the file are already splitted when it was practical to do so.
\subsubsection{14}
Only one line(278) exceeds 120 chars and it is 121 chars long.
\subsection{Wrapping Lines}
\subsubsection{15}
Line wrappings occur only after the following characters: 
\begin{verbatim}
>,=.+
\end{verbatim}
In lines 640 and 649 lines are broken before a parentheses in order to have a higher-level break so the convention is respected.
\begin{lstlisting}
                            if (resourcesUtil.filterConnectorResources
                                    (resourceManager.getAllResources(), moduleName, true).size() > 0) {
\end{lstlisting}
\subsubsection{16}
Higher level breaks are used.
\subsubsection{17}
Alignment of new statements is correct.
\subsection{Comments}
\subsubsection{18}
The assigned class lacks of documentation. Most of the methods are not documented, and even if some comments to explain what code does are present, they are very few and not present in a constant way but only in some parts of the code.
\subsubsection{19}
Line 594 is the only one that has commented-out code and it has a justification to have the code, but not why it is commented out and when it should be put back in the code or removed.

\begin{lstlisting}
        if (/*env.isDas() && */ Deployment.UNDEPLOYMENT_VALIDATION.equals(event.type())) {
\end{lstlisting}
\subsection{Java Source Files}
\subsubsection{20}
The file ConnectorDeployer.java contains only the ConnectorDeployer public class.
\subsubsection{21}
The ConnectorDeployer public class is the first and only class in the file.
\subsubsection{22}
Since the following methods have no documentation this check is not doable.
\begin{itemize}
	\item event (due to implementation of EventListener)
	\item preDestroy (due to implementation of PreDestroy)
\end{itemize}
The other methods are coherent with documentation.
\subsubsection{23} 
As stated in the point above, two public methods are not documented.
Also most of the private methods have no documentation, and it should be provided.
\subsection{Package and Import Statements}
\subsubsection{24}
The package statement package com.sun.enterprise.connectors.module; is
the first non-comment statement and is followed by the import statements.
\subsection{Class and Interface Declarations}
\subsubsection{25}
\paragraph{A}
The class has a documentation comment
\paragraph{B}
The class statement follows his doc
\paragraph{C}
There is no implementation comment
\paragraph{D}
Static variables are after the instance variables, so the convention is not respected.
\paragraph{E}
There are only private variables
\paragraph{F}
There is the empty constructor
\paragraph{G}
The methods follow the constructor
\subsubsection{26}
Methods are grouped by functionality.
\subsubsection{27}
Code is free of duplicates longer than one line.
\subsection{Initialization and Declarations}
\subsubsection{28}
Variables and class members are of the correct type and they have the right visibility.
\subsubsection{29}
Variables are declared in the proper scope.
\subsubsection{30}
Constructors are always called unless it is intentional to assign ``null'' to the variable. %TODO ma tipo quando mi ritorna un oggetto da un metodo di un altro oggetto � da considerare come una chiamata al costruttore? 
%affermativo
\subsubsection{31}
classLoader in the ``load'' method is probably intentionally left null, depending on the computation that follows its declaration. (line 176)
All the other variables are initialized before use.
\subsubsection{32}
Every variable is initialized where it is declared, and if it requires further computation to be assigned with a value it is explicitly initialized with null, which sometimes is a valid value (see 31), and later computed and assigned.
\subsubsection{33}
Declarations appear at the beginning of blocks except for the ``event'' method, in which they are declared based on context in which they are used. 
%ENDROBSECTION (this is used to automatic todo checks
\subsection{Method Calls}
\subsubsection{34}
%TODO qui manca la documentazione di un sacco di metodi, le scrivo che non si trova e che quindi non posso controllare? devo anche specificare di quali metodi manca la documentazione? %Di' che manca per certi metodi e listali brvemente
\subsubsection{35}
%TODO macosacazzovuoldire?
%Che il programmatore non si sia confuso e abbia chiamato qualcosa come ``suckBallz'' invece che ``suckMyBallz''
\subsubsection{36}
%TODO per properly intende che vengono usate per qualcosa e non rimangono inutilizzate? E che non vengano usate ad esempio aspettandosi che ritornino ``this`` quando invece sono void.
\subsection{Arrays}
\subsubsection{37}
No issues were found regarding array indexing. All arrays and lists are accessed 
either with an enhanced for or in a while loop with an iterator that starts from the
first element and scans all the other elements until there are no more.
\subsubsection{38}
As explained in the previous point, no issues were found.
\subsubsection{39}
%TODO domanda: in caso di array o liste ritornate da metodi?
%Mah, credo in generale, se viene creato un nuovo elemento di un array non viene solo dimensionato l'array ma anche creato un ``new'' element.
\subsection{Object Comparison}
\subsubsection{40}
%TODO tutti gli oggetti che vengono confrontati con null usano !=, secondo te devo segnalarlo o non li conto? io credo di no
%null non si pu� confrontare con ``equals''
No issues were found regarding object comparisons as '==' is never used.
\subsection{Output Format}
\subsubsection{41}
The only outputs of the methods that were assigned to us are the
entries logged in the logger and they're all free of spelling
and grammatical errors.
\subsubsection{42}
%TODO domanda: nei nostri messaggi di errore dei logger non mi sembra ci siano grandi informazioni su come correggere l'errore a parte lo stacktrace dell'eccezione e poche altre parole, tu che dici?
%hai ragione, ma � molto comune, che facciamo? vuoi mettere di essere pi� esplicativi o linkare a una pagina risolutiva?
\subsubsection{43}
%TODO domanda: questa proprio boh....
%Eh, semplicemente controlla che il logging non sia fatto con tabulazioni a random o colori strani.
\subsection{Computation, Comparisons and Assignments}
\subsubsection{44}
No examples of brutish programming have been found.
\subsubsection{45}
%TODO order of computation/evaluation?
\subsubsection{46}
%TODO intende che le espressioni devono essere ben parentesizzate tipo se devo fare un + prima di un *?
%Si, e che non siano invece usate laddove sono inutili
\subsubsection{47}
\subsubsection{48}
\subsubsection{49}
%TODO corretti in che senso? O.O
%credo nel senso che non siano sbagliati, ovvero che siano le condizioni giuste
\subsubsection{50}
%TODO uhm boh...
%Che non siano catchate eccezioni che non possono essere sollevate, o di non avere una eccezione pi� generica catchata prima di una pi� dettagliata (nel qual caso hai del codice non raggiungibile)
\subsubsection{51}
%TODO maledetta lei e la sua cazzo di documentazione
%Implicit type conversion � una rogna da checkare, mi sa che ci converr� caricare il progetto in eclipse e farlo fare a lui, se vuoi ne parliamo
\subsection{Exceptions}
\subsubsection{52}
%TODO non trovo le classi nella documentazione per controllare che eccezioni lanciano....
%dopo le cerco
%TODO Erica nel metodo registerBeanValidator il primo try non ha un catch block
%si che ce l'ha
\subsubsection{53}
%TODO se quando catcho un'eccezione la loggo e basta è da ritenere un'azione appropriata?
%eh, se l'eccezione � del tipo ``connection reset'' o ``invalid file'' direi di si'
Two issues have been found regarding appropriate actions 
taken in catch blocks because, in the following extracts
from two of the methods assigned to us, no actions at all
are taken:
\begin{itemize}
	\item registerBeanValidator: \\
		\begin{lstlisting}
		try {
			if (inputStream != null) {
				inputStream.close();
			}
		} catch (Exception e) {
			// ignore ?
		}
		\end{lstlisting}
	\item getValidationMappingDescriptors: \\
		\begin{lstlisting}
		try {
			reader.close();
		} catch (Exception e) {
			//ignore ?
		}
		\end{lstlisting}
\end{itemize}
\subsection{Flow of Control}
\subsubsection{54 and 55}
There are no switch statements in the methods that were assigned to us.
\subsection{Files} %TODO questo è un boh generale T.T scusami T.T
%Vuoi che la guardi io? Dovrei avere capito cosa sia XD
\subsubsection{57}
All files are declared and opened properly.
\subsubsection{58}
All opened files are closed in a ``finally'' block. %TODO riga 515 davvero chiude tutto o solo l'ultimo?
\subsubsection{59}
Since files are read line by line with a buffered reader the condition is correctly checked with a null-check for the read line.
\subsubsection{60}
Exceptions are just caught and logged. If an exception is thrown while manipulating the file it is correctly closed.
\section{Other Highlighted Problems}
%List here all the parts of code that
%you think create or may create a bug
%and explain why
%TODO registerBeanValidator does not close all streams but only the last one, which is bad practice in general

\section{Appendix}
Appendix for Roberto Clapis\\
Work hours: 10
\begin{center}
	Software Used:\\
	\-\\
	\begin{tabular}{*{2}{c}}
		\toprule
		Task & Software \\
		\midrule
		Edit \LaTeX\ Source & Vim\\
		Convert \LaTeX\ source to PDF & \LaTeX\-MK\\
		Spell Check & Aspell \\
		\LaTeX\ Check & LaCheck\\
		Analyze brackets convenction & SonarQube\\
		Analyze naming convenctions & sed, grep, cat, tr\\
		\bottomrule
	\end{tabular}
\end{center}
\-\\
\-\\
Appendix for Erica Stella\\
Work hours: 6 @ 14:00 16/12
\begin{center}
	Software Used:\\
	\-\\
	\begin{tabular}{*{2}{c}}
		\toprule
		Task & Software \\
		\midrule
		Edit \LaTeX\ Source & TexStudio\\
		Convert \LaTeX\ source to PDF & \LaTeX\-MK\\
		\bottomrule
	\end{tabular}
\end{center}

\end{document}
